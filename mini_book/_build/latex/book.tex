%% Generated by Sphinx.
\def\sphinxdocclass{jupyterBook}
\documentclass[letterpaper,10pt,english]{jupyterBook}
\ifdefined\pdfpxdimen
   \let\sphinxpxdimen\pdfpxdimen\else\newdimen\sphinxpxdimen
\fi \sphinxpxdimen=.75bp\relax
\ifdefined\pdfimageresolution
    \pdfimageresolution= \numexpr \dimexpr1in\relax/\sphinxpxdimen\relax
\fi
%% let collapsible pdf bookmarks panel have high depth per default
\PassOptionsToPackage{bookmarksdepth=5}{hyperref}
%% turn off hyperref patch of \index as sphinx.xdy xindy module takes care of
%% suitable \hyperpage mark-up, working around hyperref-xindy incompatibility
\PassOptionsToPackage{hyperindex=false}{hyperref}
%% memoir class requires extra handling
\makeatletter\@ifclassloaded{memoir}
{\ifdefined\memhyperindexfalse\memhyperindexfalse\fi}{}\makeatother

\PassOptionsToPackage{warn}{textcomp}

%\catcode`^^^^00a0\active\protected\def^^^^00a0{\leavevmode\nobreak\ }
\usepackage{cmap}
\usepackage{fontspec}
\defaultfontfeatures[\rmfamily,\sffamily,\ttfamily]{}
\usepackage{amsmath,amssymb,amstext}
\usepackage{polyglossia}
\setmainlanguage{english}



\setmainfont{FreeSerif}[
  Extension      = .otf,
  UprightFont    = *,
  ItalicFont     = *Italic,
  BoldFont       = *Bold,
  BoldItalicFont = *BoldItalic
]
\setsansfont{FreeSans}[
  Extension      = .otf,
  UprightFont    = *,
  ItalicFont     = *Oblique,
  BoldFont       = *Bold,
  BoldItalicFont = *BoldOblique,
]
\setmonofont{FreeMono}[
  Extension      = .otf,
  UprightFont    = *,
  ItalicFont     = *Oblique,
  BoldFont       = *Bold,
  BoldItalicFont = *BoldOblique,
]



\usepackage[Bjarne]{fncychap}
\usepackage[,numfigreset=1,mathnumfig]{sphinx}

\fvset{fontsize=\small}
\usepackage{geometry}


% Include hyperref last.
\usepackage{hyperref}
% Fix anchor placement for figures with captions.
\usepackage{hypcap}% it must be loaded after hyperref.
% Set up styles of URL: it should be placed after hyperref.
\urlstyle{same}

\addto\captionsenglish{\renewcommand{\contentsname}{Part 1}}

\usepackage{sphinxmessages}



        % Start of preamble defined in sphinx-jupyterbook-latex %
         \usepackage[Latin,Greek]{ucharclasses}
        \usepackage{unicode-math}
        % fixing title of the toc
        \addto\captionsenglish{\renewcommand{\contentsname}{Contents}}
        \hypersetup{
            pdfencoding=auto,
            psdextra
        }
        % End of preamble defined in sphinx-jupyterbook-latex %
        

\title{Scientific Python QuickStart}
\date{Dec 11, 2023}
\release{}
\author{Thomas J.\@{} Sargent and John Stachurski}
\newcommand{\sphinxlogo}{\vbox{}}
\renewcommand{\releasename}{}
\makeindex
\begin{document}

\pagestyle{empty}
\sphinxmaketitle
\pagestyle{plain}
\sphinxtableofcontents
\pagestyle{normal}
\phantomsection\label{\detokenize{docs/index::doc}}


\sphinxAtStartPar
Aquí vamos a recopilar una serie de commando últiles para JupyterBooks.

\sphinxAtStartPar
\sphinxhref{https://jupyterbook.org/en/stable/intro.html}{Web de JupyterBooks}

\sphinxAtStartPar
\sphinxhref{https://myst-parser.readthedocs.io/en/latest/index.html}{Web de MyST}

\sphinxAtStartPar
\sphinxhref{https://executablebooks.org/en/latest/}{Web de ExecutableBooks}
\begin{itemize}
\item {} 
\sphinxAtStartPar
Part 1

\begin{itemize}
\item {} 
\sphinxAtStartPar
{\hyperref[\detokenize{docs/01_00_Code_Blocks_y_ecuaciones::doc}]{\sphinxcrossref{Code Blocks y ecuaciones}}}
\begin{itemize}
\item {} 
\sphinxAtStartPar
{\hyperref[\detokenize{docs/01_01_Code_Blocks::doc}]{\sphinxcrossref{Code cells y code Blocks}}}

\item {} 
\sphinxAtStartPar
{\hyperref[\detokenize{docs/01_02_Ecuaciones::doc}]{\sphinxcrossref{Ecuaciones}}}

\end{itemize}

\item {} 
\sphinxAtStartPar
{\hyperref[\detokenize{docs/02_00_Roles_and_directives::doc}]{\sphinxcrossref{Roles and directives}}}

\item {} 
\sphinxAtStartPar
{\hyperref[\detokenize{docs/03_00_referenciar_cosas::doc}]{\sphinxcrossref{Referenciar cosas}}}

\item {} 
\sphinxAtStartPar
{\hyperref[\detokenize{docs/04_00_Teoremas_pruebas_y_algoritmos::doc}]{\sphinxcrossref{Teoremas, pruebas, algoritmos …}}}

\end{itemize}
\end{itemize}

\sphinxstepscope
\begin{quote}

\sphinxAtStartPar
Dec 11, 2023 | 20 words | 0 min read
\end{quote}


\chapter{Code Blocks y ecuaciones}
\label{\detokenize{docs/01_00_Code_Blocks_y_ecuaciones:code-blocks-y-ecuaciones}}\label{\detokenize{docs/01_00_Code_Blocks_y_ecuaciones:sec-code-blocks-y-ecuaciones}}\label{\detokenize{docs/01_00_Code_Blocks_y_ecuaciones::doc}}
\sphinxAtStartPar
Veamos como escrirbir bloques de código así como ecuaciones en JupyterBook usando MyST.
\begin{itemize}
\item {} 
\sphinxAtStartPar
{\hyperref[\detokenize{docs/01_01_Code_Blocks::doc}]{\sphinxcrossref{Code cells y code Blocks}}}

\item {} 
\sphinxAtStartPar
{\hyperref[\detokenize{docs/01_02_Ecuaciones::doc}]{\sphinxcrossref{Ecuaciones}}}

\end{itemize}

\sphinxstepscope
\begin{quote}

\sphinxAtStartPar
Dec 11, 2023 | 266 words | 1 min read
\end{quote}


\section{Code cells y code Blocks}
\label{\detokenize{docs/01_01_Code_Blocks:code-cells-y-code-blocks}}\label{\detokenize{docs/01_01_Code_Blocks:sec-code-blocks}}\label{\detokenize{docs/01_01_Code_Blocks::doc}}
\begin{sphinxShadowBox}
\begin{itemize}
\item {} 
\sphinxAtStartPar
\phantomsection\label{\detokenize{docs/01_01_Code_Blocks:id1}}{\hyperref[\detokenize{docs/01_01_Code_Blocks:code-blocks-and-outputs}]{\sphinxcrossref{Code blocks and outputs}}}

\item {} 
\sphinxAtStartPar
\phantomsection\label{\detokenize{docs/01_01_Code_Blocks:id2}}{\hyperref[\detokenize{docs/01_01_Code_Blocks:glue-para-insertar-variables-en-el-texto}]{\sphinxcrossref{glue para insertar variables en el texto}}}

\item {} 
\sphinxAtStartPar
\phantomsection\label{\detokenize{docs/01_01_Code_Blocks:id3}}{\hyperref[\detokenize{docs/01_01_Code_Blocks:estadisticas-de-las-ejecuciones}]{\sphinxcrossref{Estadisticas de las ejecuciones}}}

\end{itemize}
\end{sphinxShadowBox}

\sphinxAtStartPar
En realidad no es necesario poner este indice local. Ya hay el indice de la izquierda y este te pone en azul los titulos de las secciones.


\subsection{Code blocks and outputs}
\label{\detokenize{docs/01_01_Code_Blocks:code-blocks-and-outputs}}
\sphinxAtStartPar
Los code block son un tipo de (directives){[}./Roles\_and\_directives.md{]}.

\sphinxAtStartPar
Jupyter Book will also embed your code blocks and output in your book.
For example, here’s some sample Matplotlib code:

\begin{sphinxuseclass}{cell}\begin{sphinxVerbatimInput}

\begin{sphinxuseclass}{cell_input}
\begin{sphinxVerbatim}[commandchars=\\\{\}]
\PYG{k+kn}{from} \PYG{n+nn}{matplotlib} \PYG{k+kn}{import} \PYG{n}{rcParams}\PYG{p}{,} \PYG{n}{cycler}
\PYG{k+kn}{import} \PYG{n+nn}{matplotlib}\PYG{n+nn}{.}\PYG{n+nn}{pyplot} \PYG{k}{as} \PYG{n+nn}{plt}
\PYG{k+kn}{import} \PYG{n+nn}{numpy} \PYG{k}{as} \PYG{n+nn}{np}
\PYG{n}{plt}\PYG{o}{.}\PYG{n}{ion}\PYG{p}{(}\PYG{p}{)}
\end{sphinxVerbatim}

\end{sphinxuseclass}\end{sphinxVerbatimInput}
\begin{sphinxVerbatimOutput}

\begin{sphinxuseclass}{cell_output}
\begin{sphinxVerbatim}[commandchars=\\\{\}]
\PYGZlt{}contextlib.ExitStack at 0x7f58afab7820\PYGZgt{}
\end{sphinxVerbatim}

\end{sphinxuseclass}\end{sphinxVerbatimOutput}

\end{sphinxuseclass}
\begin{sphinxuseclass}{cell}\begin{sphinxVerbatimInput}

\begin{sphinxuseclass}{cell_input}
\begin{sphinxVerbatim}[commandchars=\\\{\}]
\PYG{c+c1}{\PYGZsh{} Fixing random state for reproducibility}
\PYG{n}{np}\PYG{o}{.}\PYG{n}{random}\PYG{o}{.}\PYG{n}{seed}\PYG{p}{(}\PYG{l+m+mi}{19680801}\PYG{p}{)}

\PYG{n}{N} \PYG{o}{=} \PYG{l+m+mi}{10}
\PYG{n}{data} \PYG{o}{=} \PYG{p}{[}\PYG{n}{np}\PYG{o}{.}\PYG{n}{logspace}\PYG{p}{(}\PYG{l+m+mi}{0}\PYG{p}{,} \PYG{l+m+mi}{1}\PYG{p}{,} \PYG{l+m+mi}{100}\PYG{p}{)} \PYG{o}{+} \PYG{n}{np}\PYG{o}{.}\PYG{n}{random}\PYG{o}{.}\PYG{n}{randn}\PYG{p}{(}\PYG{l+m+mi}{100}\PYG{p}{)} \PYG{o}{+} \PYG{n}{ii} \PYG{k}{for} \PYG{n}{ii} \PYG{o+ow}{in} \PYG{n+nb}{range}\PYG{p}{(}\PYG{n}{N}\PYG{p}{)}\PYG{p}{]}
\PYG{n}{data} \PYG{o}{=} \PYG{n}{np}\PYG{o}{.}\PYG{n}{array}\PYG{p}{(}\PYG{n}{data}\PYG{p}{)}\PYG{o}{.}\PYG{n}{T}
\PYG{n}{cmap} \PYG{o}{=} \PYG{n}{plt}\PYG{o}{.}\PYG{n}{cm}\PYG{o}{.}\PYG{n}{coolwarm}
\PYG{n}{rcParams}\PYG{p}{[}\PYG{l+s+s1}{\PYGZsq{}}\PYG{l+s+s1}{axes.prop\PYGZus{}cycle}\PYG{l+s+s1}{\PYGZsq{}}\PYG{p}{]} \PYG{o}{=} \PYG{n}{cycler}\PYG{p}{(}\PYG{n}{color}\PYG{o}{=}\PYG{n}{cmap}\PYG{p}{(}\PYG{n}{np}\PYG{o}{.}\PYG{n}{linspace}\PYG{p}{(}\PYG{l+m+mi}{0}\PYG{p}{,} \PYG{l+m+mi}{1}\PYG{p}{,} \PYG{n}{N}\PYG{p}{)}\PYG{p}{)}\PYG{p}{)}


\PYG{k+kn}{from} \PYG{n+nn}{matplotlib}\PYG{n+nn}{.}\PYG{n+nn}{lines} \PYG{k+kn}{import} \PYG{n}{Line2D}
\PYG{n}{custom\PYGZus{}lines} \PYG{o}{=} \PYG{p}{[}\PYG{n}{Line2D}\PYG{p}{(}\PYG{p}{[}\PYG{l+m+mi}{0}\PYG{p}{]}\PYG{p}{,} \PYG{p}{[}\PYG{l+m+mi}{0}\PYG{p}{]}\PYG{p}{,} \PYG{n}{color}\PYG{o}{=}\PYG{n}{cmap}\PYG{p}{(}\PYG{l+m+mf}{0.}\PYG{p}{)}\PYG{p}{,} \PYG{n}{lw}\PYG{o}{=}\PYG{l+m+mi}{4}\PYG{p}{)}\PYG{p}{,}
                \PYG{n}{Line2D}\PYG{p}{(}\PYG{p}{[}\PYG{l+m+mi}{0}\PYG{p}{]}\PYG{p}{,} \PYG{p}{[}\PYG{l+m+mi}{0}\PYG{p}{]}\PYG{p}{,} \PYG{n}{color}\PYG{o}{=}\PYG{n}{cmap}\PYG{p}{(}\PYG{l+m+mf}{.5}\PYG{p}{)}\PYG{p}{,} \PYG{n}{lw}\PYG{o}{=}\PYG{l+m+mi}{4}\PYG{p}{)}\PYG{p}{,}
                \PYG{n}{Line2D}\PYG{p}{(}\PYG{p}{[}\PYG{l+m+mi}{0}\PYG{p}{]}\PYG{p}{,} \PYG{p}{[}\PYG{l+m+mi}{0}\PYG{p}{]}\PYG{p}{,} \PYG{n}{color}\PYG{o}{=}\PYG{n}{cmap}\PYG{p}{(}\PYG{l+m+mf}{1.}\PYG{p}{)}\PYG{p}{,} \PYG{n}{lw}\PYG{o}{=}\PYG{l+m+mi}{4}\PYG{p}{)}\PYG{p}{]}

\PYG{n}{fig}\PYG{p}{,} \PYG{n}{ax} \PYG{o}{=} \PYG{n}{plt}\PYG{o}{.}\PYG{n}{subplots}\PYG{p}{(}\PYG{n}{figsize}\PYG{o}{=}\PYG{p}{(}\PYG{l+m+mi}{10}\PYG{p}{,} \PYG{l+m+mi}{5}\PYG{p}{)}\PYG{p}{)}
\PYG{n}{lines} \PYG{o}{=} \PYG{n}{ax}\PYG{o}{.}\PYG{n}{plot}\PYG{p}{(}\PYG{n}{data}\PYG{p}{)}
\PYG{n}{ax}\PYG{o}{.}\PYG{n}{legend}\PYG{p}{(}\PYG{n}{custom\PYGZus{}lines}\PYG{p}{,} \PYG{p}{[}\PYG{l+s+s1}{\PYGZsq{}}\PYG{l+s+s1}{Cold}\PYG{l+s+s1}{\PYGZsq{}}\PYG{p}{,} \PYG{l+s+s1}{\PYGZsq{}}\PYG{l+s+s1}{Medium}\PYG{l+s+s1}{\PYGZsq{}}\PYG{p}{,} \PYG{l+s+s1}{\PYGZsq{}}\PYG{l+s+s1}{Hot}\PYG{l+s+s1}{\PYGZsq{}}\PYG{p}{]}\PYG{p}{)}
\end{sphinxVerbatim}

\end{sphinxuseclass}\end{sphinxVerbatimInput}
\begin{sphinxVerbatimOutput}

\begin{sphinxuseclass}{cell_output}
\begin{sphinxVerbatim}[commandchars=\\\{\}]
\PYGZlt{}matplotlib.legend.Legend at 0x7f57e4e9f940\PYGZgt{}
\end{sphinxVerbatim}

\noindent\sphinxincludegraphics{{165f042da09af3f7a5e742dd4ae058315b519fb9d19a8c2ebcea9dfedcc5d077}.png}

\end{sphinxuseclass}\end{sphinxVerbatimOutput}

\end{sphinxuseclass}

\subsubsection{Celdas no ejecutables pero numeradas y con formato}
\label{\detokenize{docs/01_01_Code_Blocks:celdas-no-ejecutables-pero-numeradas-y-con-formato}}\sphinxSetupCaptionForVerbatim{This is my
multi\sphinxhyphen{}line caption. It is \sphinxstyleemphasis{pretty nifty} ;\sphinxhyphen{})}
\def\sphinxLiteralBlockLabel{\label{\detokenize{docs/01_01_Code_Blocks:label-codeblock}}}
\fvset{hllines={, 1, 3,}}%
\begin{sphinxVerbatim}[commandchars=\\\{\},numbers=left,firstnumber=10,stepnumber=1]
\PYG{n}{a} \PYG{o}{=} \PYG{l+m+mi}{2}
\PYG{n+nb}{print}\PYG{p}{(}\PYG{l+s+s1}{\PYGZsq{}}\PYG{l+s+s1}{my 1st line}\PYG{l+s+s1}{\PYGZsq{}}\PYG{p}{)}
\PYG{n+nb}{print}\PYG{p}{(}\PYG{l+s+sa}{f}\PYG{l+s+s1}{\PYGZsq{}}\PYG{l+s+s1}{my }\PYG{l+s+si}{\PYGZob{}}\PYG{n}{a}\PYG{l+s+si}{\PYGZcb{}}\PYG{l+s+s1}{nd line}\PYG{l+s+s1}{\PYGZsq{}}\PYG{p}{)}
\end{sphinxVerbatim}
\sphinxresetverbatimhllines


\subsubsection{Celdas con salida de error}
\label{\detokenize{docs/01_01_Code_Blocks:celdas-con-salida-de-error}}
\sphinxAtStartPar
Prueba de celda que da error

\begin{sphinxuseclass}{cell}\begin{sphinxVerbatimInput}

\begin{sphinxuseclass}{cell_input}
\begin{sphinxVerbatim}[commandchars=\\\{\}]
\PYG{n+nb}{print}\PYG{p}{(}\PYG{n}{val\PYGZus{}a}\PYG{p}{)}
\end{sphinxVerbatim}

\end{sphinxuseclass}\end{sphinxVerbatimInput}
\begin{sphinxVerbatimOutput}

\begin{sphinxuseclass}{cell_output}
\begin{sphinxVerbatim}[commandchars=\\\{\}]
\PYG{g+gt}{\PYGZhy{}\PYGZhy{}\PYGZhy{}\PYGZhy{}\PYGZhy{}\PYGZhy{}\PYGZhy{}\PYGZhy{}\PYGZhy{}\PYGZhy{}\PYGZhy{}\PYGZhy{}\PYGZhy{}\PYGZhy{}\PYGZhy{}\PYGZhy{}\PYGZhy{}\PYGZhy{}\PYGZhy{}\PYGZhy{}\PYGZhy{}\PYGZhy{}\PYGZhy{}\PYGZhy{}\PYGZhy{}\PYGZhy{}\PYGZhy{}\PYGZhy{}\PYGZhy{}\PYGZhy{}\PYGZhy{}\PYGZhy{}\PYGZhy{}\PYGZhy{}\PYGZhy{}\PYGZhy{}\PYGZhy{}\PYGZhy{}\PYGZhy{}\PYGZhy{}\PYGZhy{}\PYGZhy{}\PYGZhy{}\PYGZhy{}\PYGZhy{}\PYGZhy{}\PYGZhy{}\PYGZhy{}\PYGZhy{}\PYGZhy{}\PYGZhy{}\PYGZhy{}\PYGZhy{}\PYGZhy{}\PYGZhy{}\PYGZhy{}\PYGZhy{}\PYGZhy{}\PYGZhy{}\PYGZhy{}\PYGZhy{}\PYGZhy{}\PYGZhy{}\PYGZhy{}\PYGZhy{}\PYGZhy{}\PYGZhy{}\PYGZhy{}\PYGZhy{}\PYGZhy{}\PYGZhy{}\PYGZhy{}\PYGZhy{}\PYGZhy{}\PYGZhy{}}
\PYG{n+ne}{NameError}\PYG{g+gWhitespace}{                                 }Traceback (most recent call last)
\PYG{n}{Cell} \PYG{n}{In}\PYG{p}{[}\PYG{l+m+mi}{3}\PYG{p}{]}\PYG{p}{,} \PYG{n}{line} \PYG{l+m+mi}{1}
\PYG{n+ne}{\PYGZhy{}\PYGZhy{}\PYGZhy{}\PYGZhy{}\PYGZgt{} }\PYG{l+m+mi}{1} \PYG{n+nb}{print}\PYG{p}{(}\PYG{n}{val\PYGZus{}a}\PYG{p}{)}

\PYG{n+ne}{NameError}: name \PYGZsq{}val\PYGZus{}a\PYGZsq{} is not defined
\end{sphinxVerbatim}

\end{sphinxuseclass}\end{sphinxVerbatimOutput}

\end{sphinxuseclass}

\subsubsection{Celdas desplegables}
\label{\detokenize{docs/01_01_Code_Blocks:celdas-desplegables}}
\sphinxAtStartPar
Veamos una celda desplegable:

\begin{sphinxuseclass}{cell}\begin{sphinxVerbatimInput}

\begin{sphinxuseclass}{cell_input}
\begin{sphinxVerbatim}[commandchars=\\\{\}]
\PYG{k+kn}{import} \PYG{n+nn}{numpy} \PYG{k}{as} \PYG{n+nn}{np}
\PYG{k+kn}{import} \PYG{n+nn}{pandas} \PYG{k}{as} \PYG{n+nn}{pd}

\PYG{n}{np}\PYG{o}{.}\PYG{n}{random}\PYG{o}{.}\PYG{n}{seed}\PYG{p}{(}\PYG{l+m+mi}{24}\PYG{p}{)}
\PYG{n}{df} \PYG{o}{=} \PYG{n}{pd}\PYG{o}{.}\PYG{n}{DataFrame}\PYG{p}{(}\PYG{p}{\PYGZob{}}\PYG{l+s+s1}{\PYGZsq{}}\PYG{l+s+s1}{A}\PYG{l+s+s1}{\PYGZsq{}}\PYG{p}{:} \PYG{n}{np}\PYG{o}{.}\PYG{n}{linspace}\PYG{p}{(}\PYG{l+m+mi}{1}\PYG{p}{,} \PYG{l+m+mi}{10}\PYG{p}{,} \PYG{l+m+mi}{10}\PYG{p}{)}\PYG{p}{\PYGZcb{}}\PYG{p}{)}
\PYG{n}{df} \PYG{o}{=} \PYG{n}{pd}\PYG{o}{.}\PYG{n}{concat}\PYG{p}{(}\PYG{p}{[}\PYG{n}{df}\PYG{p}{,} \PYG{n}{pd}\PYG{o}{.}\PYG{n}{DataFrame}\PYG{p}{(}\PYG{n}{np}\PYG{o}{.}\PYG{n}{random}\PYG{o}{.}\PYG{n}{randn}\PYG{p}{(}\PYG{l+m+mi}{10}\PYG{p}{,} \PYG{l+m+mi}{4}\PYG{p}{)}\PYG{p}{,} \PYG{n}{columns}\PYG{o}{=}\PYG{n+nb}{list}\PYG{p}{(}\PYG{l+s+s1}{\PYGZsq{}}\PYG{l+s+s1}{BCDE}\PYG{l+s+s1}{\PYGZsq{}}\PYG{p}{)}\PYG{p}{)}\PYG{p}{]}\PYG{p}{,}
            \PYG{n}{axis}\PYG{o}{=}\PYG{l+m+mi}{1}\PYG{p}{)}
\PYG{n}{df}\PYG{o}{.}\PYG{n}{iloc}\PYG{p}{[}\PYG{l+m+mi}{3}\PYG{p}{,} \PYG{l+m+mi}{3}\PYG{p}{]} \PYG{o}{=} \PYG{n}{np}\PYG{o}{.}\PYG{n}{nan}
\PYG{n}{df}\PYG{o}{.}\PYG{n}{iloc}\PYG{p}{[}\PYG{l+m+mi}{0}\PYG{p}{,} \PYG{l+m+mi}{2}\PYG{p}{]} \PYG{o}{=} \PYG{n}{np}\PYG{o}{.}\PYG{n}{nan}

\PYG{k}{def} \PYG{n+nf}{color\PYGZus{}negative\PYGZus{}red}\PYG{p}{(}\PYG{n}{val}\PYG{p}{)}\PYG{p}{:}
    \PYG{l+s+sd}{\PYGZdq{}\PYGZdq{}\PYGZdq{}}
\PYG{l+s+sd}{    Takes a scalar and returns a string with}
\PYG{l+s+sd}{    the css property `\PYGZsq{}color: red\PYGZsq{}` for negative}
\PYG{l+s+sd}{    strings, black otherwise.}
\PYG{l+s+sd}{    \PYGZdq{}\PYGZdq{}\PYGZdq{}}
    \PYG{n}{color} \PYG{o}{=} \PYG{l+s+s1}{\PYGZsq{}}\PYG{l+s+s1}{red}\PYG{l+s+s1}{\PYGZsq{}} \PYG{k}{if} \PYG{n}{val} \PYG{o}{\PYGZlt{}} \PYG{l+m+mi}{0} \PYG{k}{else} \PYG{l+s+s1}{\PYGZsq{}}\PYG{l+s+s1}{black}\PYG{l+s+s1}{\PYGZsq{}}
    \PYG{k}{return} \PYG{l+s+s1}{\PYGZsq{}}\PYG{l+s+s1}{color: }\PYG{l+s+si}{\PYGZpc{}s}\PYG{l+s+s1}{\PYGZsq{}} \PYG{o}{\PYGZpc{}} \PYG{n}{color}

\PYG{k}{def} \PYG{n+nf}{highlight\PYGZus{}max}\PYG{p}{(}\PYG{n}{s}\PYG{p}{)}\PYG{p}{:}
    \PYG{l+s+sd}{\PYGZsq{}\PYGZsq{}\PYGZsq{}}
\PYG{l+s+sd}{    highlight the maximum in a Series yellow.}
\PYG{l+s+sd}{    \PYGZsq{}\PYGZsq{}\PYGZsq{}}
    \PYG{n}{is\PYGZus{}max} \PYG{o}{=} \PYG{n}{s} \PYG{o}{==} \PYG{n}{s}\PYG{o}{.}\PYG{n}{max}\PYG{p}{(}\PYG{p}{)}
    \PYG{k}{return} \PYG{p}{[}\PYG{l+s+s1}{\PYGZsq{}}\PYG{l+s+s1}{background\PYGZhy{}color: yellow}\PYG{l+s+s1}{\PYGZsq{}} \PYG{k}{if} \PYG{n}{v} \PYG{k}{else} \PYG{l+s+s1}{\PYGZsq{}}\PYG{l+s+s1}{\PYGZsq{}} \PYG{k}{for} \PYG{n}{v} \PYG{o+ow}{in} \PYG{n}{is\PYGZus{}max}\PYG{p}{]}

\PYG{n}{df}\PYG{o}{.}\PYG{n}{style}\PYG{o}{.}\PYGZbs{}
    \PYG{n}{applymap}\PYG{p}{(}\PYG{n}{color\PYGZus{}negative\PYGZus{}red}\PYG{p}{)}\PYG{o}{.}\PYGZbs{}
    \PYG{n}{apply}\PYG{p}{(}\PYG{n}{highlight\PYGZus{}max}\PYG{p}{)}\PYG{o}{.}\PYGZbs{}
    \PYG{n}{set\PYGZus{}table\PYGZus{}attributes}\PYG{p}{(}\PYG{l+s+s1}{\PYGZsq{}}\PYG{l+s+s1}{style=}\PYG{l+s+s1}{\PYGZdq{}}\PYG{l+s+s1}{font\PYGZhy{}size: 10px}\PYG{l+s+s1}{\PYGZdq{}}\PYG{l+s+s1}{\PYGZsq{}}\PYG{p}{)}
\end{sphinxVerbatim}

\end{sphinxuseclass}\end{sphinxVerbatimInput}
\begin{sphinxVerbatimOutput}

\begin{sphinxuseclass}{cell_output}
\begin{sphinxVerbatim}[commandchars=\\\{\}]
\PYGZlt{}pandas.io.formats.style.Styler at 0x7f57d0d84f70\PYGZgt{}
\end{sphinxVerbatim}

\end{sphinxuseclass}\end{sphinxVerbatimOutput}

\end{sphinxuseclass}

\subsubsection{Celdas con scroll}
\label{\detokenize{docs/01_01_Code_Blocks:celdas-con-scroll}}
\sphinxAtStartPar
Veamos una celca con scroll en la salida

\begin{sphinxuseclass}{cell}
\begin{sphinxuseclass}{tag_output_scroll}\begin{sphinxVerbatimInput}

\begin{sphinxuseclass}{cell_input}
\begin{sphinxVerbatim}[commandchars=\\\{\}]
\PYG{k}{for} \PYG{n}{ii} \PYG{o+ow}{in} \PYG{n+nb}{range}\PYG{p}{(}\PYG{l+m+mi}{40}\PYG{p}{)}\PYG{p}{:}
    \PYG{n+nb}{print}\PYG{p}{(}\PYG{l+s+sa}{f}\PYG{l+s+s2}{\PYGZdq{}}\PYG{l+s+s2}{this is output line }\PYG{l+s+si}{\PYGZob{}}\PYG{n}{ii}\PYG{l+s+si}{\PYGZcb{}}\PYG{l+s+s2}{\PYGZdq{}}\PYG{p}{)}
\end{sphinxVerbatim}

\end{sphinxuseclass}\end{sphinxVerbatimInput}
\begin{sphinxVerbatimOutput}

\begin{sphinxuseclass}{cell_output}
\begin{sphinxVerbatim}[commandchars=\\\{\}]
this is output line 0
this is output line 1
this is output line 2
this is output line 3
this is output line 4
this is output line 5
this is output line 6
this is output line 7
this is output line 8
this is output line 9
this is output line 10
this is output line 11
this is output line 12
this is output line 13
this is output line 14
this is output line 15
this is output line 16
this is output line 17
this is output line 18
this is output line 19
this is output line 20
this is output line 21
this is output line 22
this is output line 23
this is output line 24
this is output line 25
this is output line 26
this is output line 27
this is output line 28
this is output line 29
this is output line 30
this is output line 31
this is output line 32
this is output line 33
this is output line 34
this is output line 35
this is output line 36
this is output line 37
this is output line 38
this is output line 39
\end{sphinxVerbatim}

\end{sphinxuseclass}\end{sphinxVerbatimOutput}

\end{sphinxuseclass}
\end{sphinxuseclass}

\subsubsection{Colores en los print}
\label{\detokenize{docs/01_01_Code_Blocks:colores-en-los-print}}
\sphinxAtStartPar
Veamos ahora los colores que podemos poner en los print de python:

\begin{sphinxuseclass}{cell}\begin{sphinxVerbatimInput}

\begin{sphinxuseclass}{cell_input}
\begin{sphinxVerbatim}[commandchars=\\\{\}]
\PYG{n}{text} \PYG{o}{=} \PYG{l+s+s2}{\PYGZdq{}}\PYG{l+s+s2}{ XYZ }\PYG{l+s+s2}{\PYGZdq{}}
\PYG{n}{formatstring} \PYG{o}{=} \PYG{l+s+s2}{\PYGZdq{}}\PYG{l+s+se}{\PYGZbs{}x1b}\PYG{l+s+s2}{[}\PYG{l+s+si}{\PYGZob{}\PYGZcb{}}\PYG{l+s+s2}{m}\PYG{l+s+s2}{\PYGZdq{}} \PYG{o}{+} \PYG{n}{text} \PYG{o}{+} \PYG{l+s+s2}{\PYGZdq{}}\PYG{l+s+se}{\PYGZbs{}x1b}\PYG{l+s+s2}{[m}\PYG{l+s+s2}{\PYGZdq{}}

\PYG{n+nb}{print}\PYG{p}{(}
    \PYG{l+s+s2}{\PYGZdq{}}\PYG{l+s+s2}{ }\PYG{l+s+s2}{\PYGZdq{}} \PYG{o}{*} \PYG{l+m+mi}{6}
    \PYG{o}{+} \PYG{l+s+s2}{\PYGZdq{}}\PYG{l+s+s2}{ }\PYG{l+s+s2}{\PYGZdq{}} \PYG{o}{*} \PYG{n+nb}{len}\PYG{p}{(}\PYG{n}{text}\PYG{p}{)}
    \PYG{o}{+} \PYG{l+s+s2}{\PYGZdq{}}\PYG{l+s+s2}{\PYGZdq{}}\PYG{o}{.}\PYG{n}{join}\PYG{p}{(}\PYG{l+s+s2}{\PYGZdq{}}\PYG{l+s+s2}{\PYGZob{}}\PYG{l+s+s2}{:\PYGZca{}}\PYG{l+s+si}{\PYGZob{}\PYGZcb{}}\PYG{l+s+s2}{\PYGZcb{}}\PYG{l+s+s2}{\PYGZdq{}}\PYG{o}{.}\PYG{n}{format}\PYG{p}{(}\PYG{n}{bg}\PYG{p}{,} \PYG{n+nb}{len}\PYG{p}{(}\PYG{n}{text}\PYG{p}{)}\PYG{p}{)} \PYG{k}{for} \PYG{n}{bg} \PYG{o+ow}{in} \PYG{n+nb}{range}\PYG{p}{(}\PYG{l+m+mi}{40}\PYG{p}{,} \PYG{l+m+mi}{48}\PYG{p}{)}\PYG{p}{)}
\PYG{p}{)}
\PYG{k}{for} \PYG{n}{fg} \PYG{o+ow}{in} \PYG{n+nb}{range}\PYG{p}{(}\PYG{l+m+mi}{30}\PYG{p}{,} \PYG{l+m+mi}{38}\PYG{p}{)}\PYG{p}{:}
    \PYG{k}{for} \PYG{n}{bold} \PYG{o+ow}{in} \PYG{k+kc}{False}\PYG{p}{,} \PYG{k+kc}{True}\PYG{p}{:}
        \PYG{n}{fg\PYGZus{}code} \PYG{o}{=} \PYG{p}{(}\PYG{l+s+s2}{\PYGZdq{}}\PYG{l+s+s2}{1;}\PYG{l+s+s2}{\PYGZdq{}} \PYG{k}{if} \PYG{n}{bold} \PYG{k}{else} \PYG{l+s+s2}{\PYGZdq{}}\PYG{l+s+s2}{\PYGZdq{}}\PYG{p}{)} \PYG{o}{+} \PYG{n+nb}{str}\PYG{p}{(}\PYG{n}{fg}\PYG{p}{)}
        \PYG{n+nb}{print}\PYG{p}{(}
            \PYG{l+s+s2}{\PYGZdq{}}\PYG{l+s+s2}{ }\PYG{l+s+si}{\PYGZob{}:\PYGZgt{}4\PYGZcb{}}\PYG{l+s+s2}{ }\PYG{l+s+s2}{\PYGZdq{}}\PYG{o}{.}\PYG{n}{format}\PYG{p}{(}\PYG{n}{fg\PYGZus{}code}\PYG{p}{)}
            \PYG{o}{+} \PYG{n}{formatstring}\PYG{o}{.}\PYG{n}{format}\PYG{p}{(}\PYG{n}{fg\PYGZus{}code}\PYG{p}{)}
            \PYG{o}{+} \PYG{l+s+s2}{\PYGZdq{}}\PYG{l+s+s2}{\PYGZdq{}}\PYG{o}{.}\PYG{n}{join}\PYG{p}{(}
                \PYG{n}{formatstring}\PYG{o}{.}\PYG{n}{format}\PYG{p}{(}\PYG{n}{fg\PYGZus{}code} \PYG{o}{+} \PYG{l+s+s2}{\PYGZdq{}}\PYG{l+s+s2}{;}\PYG{l+s+s2}{\PYGZdq{}} \PYG{o}{+} \PYG{n+nb}{str}\PYG{p}{(}\PYG{n}{bg}\PYG{p}{)}\PYG{p}{)} \PYG{k}{for} \PYG{n}{bg} \PYG{o+ow}{in} \PYG{n+nb}{range}\PYG{p}{(}\PYG{l+m+mi}{40}\PYG{p}{,} \PYG{l+m+mi}{48}\PYG{p}{)}
            \PYG{p}{)}
        \PYG{p}{)}
\end{sphinxVerbatim}

\end{sphinxuseclass}\end{sphinxVerbatimInput}
\begin{sphinxVerbatimOutput}

\begin{sphinxuseclass}{cell_output}
\begin{sphinxVerbatim}[commandchars=\\\{\}]
            40   41   42   43   44   45   46   47  
   30 \PYG{Color+ColorBlack}{ XYZ }\PYG{Color+ColorBlack+ColorBlackBGBlack}{ XYZ }\PYG{Color+ColorBlack+ColorBlackBGRed}{ XYZ }\PYG{Color+ColorBlack+ColorBlackBGGreen}{ XYZ }\PYG{Color+ColorBlack+ColorBlackBGYellow}{ XYZ }\PYG{Color+ColorBlack+ColorBlackBGBlue}{ XYZ }\PYG{Color+ColorBlack+ColorBlackBGMagenta}{ XYZ }\PYG{Color+ColorBlack+ColorBlackBGCyan}{ XYZ }\PYG{Color+ColorBlack+ColorBlackBGWhite}{ XYZ }
 1;30 \PYG{Color+ColorBold+ColorBoldBlack}{ XYZ }\PYG{Color+ColorBold+ColorBoldBlack+ColorBoldBlackBGBlack}{ XYZ }\PYG{Color+ColorBold+ColorBoldBlack+ColorBoldBlackBGRed}{ XYZ }\PYG{Color+ColorBold+ColorBoldBlack+ColorBoldBlackBGGreen}{ XYZ }\PYG{Color+ColorBold+ColorBoldBlack+ColorBoldBlackBGYellow}{ XYZ }\PYG{Color+ColorBold+ColorBoldBlack+ColorBoldBlackBGBlue}{ XYZ }\PYG{Color+ColorBold+ColorBoldBlack+ColorBoldBlackBGMagenta}{ XYZ }\PYG{Color+ColorBold+ColorBoldBlack+ColorBoldBlackBGCyan}{ XYZ }\PYG{Color+ColorBold+ColorBoldBlack+ColorBoldBlackBGWhite}{ XYZ }
   31 \PYG{Color+ColorRed}{ XYZ }\PYG{Color+ColorRed+ColorRedBGBlack}{ XYZ }\PYG{Color+ColorRed+ColorRedBGRed}{ XYZ }\PYG{Color+ColorRed+ColorRedBGGreen}{ XYZ }\PYG{Color+ColorRed+ColorRedBGYellow}{ XYZ }\PYG{Color+ColorRed+ColorRedBGBlue}{ XYZ }\PYG{Color+ColorRed+ColorRedBGMagenta}{ XYZ }\PYG{Color+ColorRed+ColorRedBGCyan}{ XYZ }\PYG{Color+ColorRed+ColorRedBGWhite}{ XYZ }
 1;31 \PYG{Color+ColorBold+ColorBoldRed}{ XYZ }\PYG{Color+ColorBold+ColorBoldRed+ColorBoldRedBGBlack}{ XYZ }\PYG{Color+ColorBold+ColorBoldRed+ColorBoldRedBGRed}{ XYZ }\PYG{Color+ColorBold+ColorBoldRed+ColorBoldRedBGGreen}{ XYZ }\PYG{Color+ColorBold+ColorBoldRed+ColorBoldRedBGYellow}{ XYZ }\PYG{Color+ColorBold+ColorBoldRed+ColorBoldRedBGBlue}{ XYZ }\PYG{Color+ColorBold+ColorBoldRed+ColorBoldRedBGMagenta}{ XYZ }\PYG{Color+ColorBold+ColorBoldRed+ColorBoldRedBGCyan}{ XYZ }\PYG{Color+ColorBold+ColorBoldRed+ColorBoldRedBGWhite}{ XYZ }
   32 \PYG{Color+ColorGreen}{ XYZ }\PYG{Color+ColorGreen+ColorGreenBGBlack}{ XYZ }\PYG{Color+ColorGreen+ColorGreenBGRed}{ XYZ }\PYG{Color+ColorGreen+ColorGreenBGGreen}{ XYZ }\PYG{Color+ColorGreen+ColorGreenBGYellow}{ XYZ }\PYG{Color+ColorGreen+ColorGreenBGBlue}{ XYZ }\PYG{Color+ColorGreen+ColorGreenBGMagenta}{ XYZ }\PYG{Color+ColorGreen+ColorGreenBGCyan}{ XYZ }\PYG{Color+ColorGreen+ColorGreenBGWhite}{ XYZ }
 1;32 \PYG{Color+ColorBold+ColorBoldGreen}{ XYZ }\PYG{Color+ColorBold+ColorBoldGreen+ColorBoldGreenBGBlack}{ XYZ }\PYG{Color+ColorBold+ColorBoldGreen+ColorBoldGreenBGRed}{ XYZ }\PYG{Color+ColorBold+ColorBoldGreen+ColorBoldGreenBGGreen}{ XYZ }\PYG{Color+ColorBold+ColorBoldGreen+ColorBoldGreenBGYellow}{ XYZ }\PYG{Color+ColorBold+ColorBoldGreen+ColorBoldGreenBGBlue}{ XYZ }\PYG{Color+ColorBold+ColorBoldGreen+ColorBoldGreenBGMagenta}{ XYZ }\PYG{Color+ColorBold+ColorBoldGreen+ColorBoldGreenBGCyan}{ XYZ }\PYG{Color+ColorBold+ColorBoldGreen+ColorBoldGreenBGWhite}{ XYZ }
   33 \PYG{Color+ColorYellow}{ XYZ }\PYG{Color+ColorYellow+ColorYellowBGBlack}{ XYZ }\PYG{Color+ColorYellow+ColorYellowBGRed}{ XYZ }\PYG{Color+ColorYellow+ColorYellowBGGreen}{ XYZ }\PYG{Color+ColorYellow+ColorYellowBGYellow}{ XYZ }\PYG{Color+ColorYellow+ColorYellowBGBlue}{ XYZ }\PYG{Color+ColorYellow+ColorYellowBGMagenta}{ XYZ }\PYG{Color+ColorYellow+ColorYellowBGCyan}{ XYZ }\PYG{Color+ColorYellow+ColorYellowBGWhite}{ XYZ }
 1;33 \PYG{Color+ColorBold+ColorBoldYellow}{ XYZ }\PYG{Color+ColorBold+ColorBoldYellow+ColorBoldYellowBGBlack}{ XYZ }\PYG{Color+ColorBold+ColorBoldYellow+ColorBoldYellowBGRed}{ XYZ }\PYG{Color+ColorBold+ColorBoldYellow+ColorBoldYellowBGGreen}{ XYZ }\PYG{Color+ColorBold+ColorBoldYellow+ColorBoldYellowBGYellow}{ XYZ }\PYG{Color+ColorBold+ColorBoldYellow+ColorBoldYellowBGBlue}{ XYZ }\PYG{Color+ColorBold+ColorBoldYellow+ColorBoldYellowBGMagenta}{ XYZ }\PYG{Color+ColorBold+ColorBoldYellow+ColorBoldYellowBGCyan}{ XYZ }\PYG{Color+ColorBold+ColorBoldYellow+ColorBoldYellowBGWhite}{ XYZ }
   34 \PYG{Color+ColorBlue}{ XYZ }\PYG{Color+ColorBlue+ColorBlueBGBlack}{ XYZ }\PYG{Color+ColorBlue+ColorBlueBGRed}{ XYZ }\PYG{Color+ColorBlue+ColorBlueBGGreen}{ XYZ }\PYG{Color+ColorBlue+ColorBlueBGYellow}{ XYZ }\PYG{Color+ColorBlue+ColorBlueBGBlue}{ XYZ }\PYG{Color+ColorBlue+ColorBlueBGMagenta}{ XYZ }\PYG{Color+ColorBlue+ColorBlueBGCyan}{ XYZ }\PYG{Color+ColorBlue+ColorBlueBGWhite}{ XYZ }
 1;34 \PYG{Color+ColorBold+ColorBoldBlue}{ XYZ }\PYG{Color+ColorBold+ColorBoldBlue+ColorBoldBlueBGBlack}{ XYZ }\PYG{Color+ColorBold+ColorBoldBlue+ColorBoldBlueBGRed}{ XYZ }\PYG{Color+ColorBold+ColorBoldBlue+ColorBoldBlueBGGreen}{ XYZ }\PYG{Color+ColorBold+ColorBoldBlue+ColorBoldBlueBGYellow}{ XYZ }\PYG{Color+ColorBold+ColorBoldBlue+ColorBoldBlueBGBlue}{ XYZ }\PYG{Color+ColorBold+ColorBoldBlue+ColorBoldBlueBGMagenta}{ XYZ }\PYG{Color+ColorBold+ColorBoldBlue+ColorBoldBlueBGCyan}{ XYZ }\PYG{Color+ColorBold+ColorBoldBlue+ColorBoldBlueBGWhite}{ XYZ }
   35 \PYG{Color+ColorMagenta}{ XYZ }\PYG{Color+ColorMagenta+ColorMagentaBGBlack}{ XYZ }\PYG{Color+ColorMagenta+ColorMagentaBGRed}{ XYZ }\PYG{Color+ColorMagenta+ColorMagentaBGGreen}{ XYZ }\PYG{Color+ColorMagenta+ColorMagentaBGYellow}{ XYZ }\PYG{Color+ColorMagenta+ColorMagentaBGBlue}{ XYZ }\PYG{Color+ColorMagenta+ColorMagentaBGMagenta}{ XYZ }\PYG{Color+ColorMagenta+ColorMagentaBGCyan}{ XYZ }\PYG{Color+ColorMagenta+ColorMagentaBGWhite}{ XYZ }
 1;35 \PYG{Color+ColorBold+ColorBoldMagenta}{ XYZ }\PYG{Color+ColorBold+ColorBoldMagenta+ColorBoldMagentaBGBlack}{ XYZ }\PYG{Color+ColorBold+ColorBoldMagenta+ColorBoldMagentaBGRed}{ XYZ }\PYG{Color+ColorBold+ColorBoldMagenta+ColorBoldMagentaBGGreen}{ XYZ }\PYG{Color+ColorBold+ColorBoldMagenta+ColorBoldMagentaBGYellow}{ XYZ }\PYG{Color+ColorBold+ColorBoldMagenta+ColorBoldMagentaBGBlue}{ XYZ }\PYG{Color+ColorBold+ColorBoldMagenta+ColorBoldMagentaBGMagenta}{ XYZ }\PYG{Color+ColorBold+ColorBoldMagenta+ColorBoldMagentaBGCyan}{ XYZ }\PYG{Color+ColorBold+ColorBoldMagenta+ColorBoldMagentaBGWhite}{ XYZ }
   36 \PYG{Color+ColorCyan}{ XYZ }\PYG{Color+ColorCyan+ColorCyanBGBlack}{ XYZ }\PYG{Color+ColorCyan+ColorCyanBGRed}{ XYZ }\PYG{Color+ColorCyan+ColorCyanBGGreen}{ XYZ }\PYG{Color+ColorCyan+ColorCyanBGYellow}{ XYZ }\PYG{Color+ColorCyan+ColorCyanBGBlue}{ XYZ }\PYG{Color+ColorCyan+ColorCyanBGMagenta}{ XYZ }\PYG{Color+ColorCyan+ColorCyanBGCyan}{ XYZ }\PYG{Color+ColorCyan+ColorCyanBGWhite}{ XYZ }
 1;36 \PYG{Color+ColorBold+ColorBoldCyan}{ XYZ }\PYG{Color+ColorBold+ColorBoldCyan+ColorBoldCyanBGBlack}{ XYZ }\PYG{Color+ColorBold+ColorBoldCyan+ColorBoldCyanBGRed}{ XYZ }\PYG{Color+ColorBold+ColorBoldCyan+ColorBoldCyanBGGreen}{ XYZ }\PYG{Color+ColorBold+ColorBoldCyan+ColorBoldCyanBGYellow}{ XYZ }\PYG{Color+ColorBold+ColorBoldCyan+ColorBoldCyanBGBlue}{ XYZ }\PYG{Color+ColorBold+ColorBoldCyan+ColorBoldCyanBGMagenta}{ XYZ }\PYG{Color+ColorBold+ColorBoldCyan+ColorBoldCyanBGCyan}{ XYZ }\PYG{Color+ColorBold+ColorBoldCyan+ColorBoldCyanBGWhite}{ XYZ }
   37 \PYG{Color+ColorWhite}{ XYZ }\PYG{Color+ColorWhite+ColorWhiteBGBlack}{ XYZ }\PYG{Color+ColorWhite+ColorWhiteBGRed}{ XYZ }\PYG{Color+ColorWhite+ColorWhiteBGGreen}{ XYZ }\PYG{Color+ColorWhite+ColorWhiteBGYellow}{ XYZ }\PYG{Color+ColorWhite+ColorWhiteBGBlue}{ XYZ }\PYG{Color+ColorWhite+ColorWhiteBGMagenta}{ XYZ }\PYG{Color+ColorWhite+ColorWhiteBGCyan}{ XYZ }\PYG{Color+ColorWhite+ColorWhiteBGWhite}{ XYZ }
 1;37 \PYG{Color+ColorBold+ColorBoldWhite}{ XYZ }\PYG{Color+ColorBold+ColorBoldWhite+ColorBoldWhiteBGBlack}{ XYZ }\PYG{Color+ColorBold+ColorBoldWhite+ColorBoldWhiteBGRed}{ XYZ }\PYG{Color+ColorBold+ColorBoldWhite+ColorBoldWhiteBGGreen}{ XYZ }\PYG{Color+ColorBold+ColorBoldWhite+ColorBoldWhiteBGYellow}{ XYZ }\PYG{Color+ColorBold+ColorBoldWhite+ColorBoldWhiteBGBlue}{ XYZ }\PYG{Color+ColorBold+ColorBoldWhite+ColorBoldWhiteBGMagenta}{ XYZ }\PYG{Color+ColorBold+ColorBoldWhite+ColorBoldWhiteBGCyan}{ XYZ }\PYG{Color+ColorBold+ColorBoldWhite+ColorBoldWhiteBGWhite}{ XYZ }
\end{sphinxVerbatim}

\end{sphinxuseclass}\end{sphinxVerbatimOutput}

\end{sphinxuseclass}

\subsection{glue para insertar variables en el texto}
\label{\detokenize{docs/01_01_Code_Blocks:glue-para-insertar-variables-en-el-texto}}

\subsubsection{“Gluing” variables en el notebook}
\label{\detokenize{docs/01_01_Code_Blocks:gluing-variables-en-el-notebook}}
\sphinxAtStartPar
Tenemos que importar la función \sphinxcode{\sphinxupquote{glue()}} de la libreria \sphinxcode{\sphinxupquote{myst\_nb}}:

\begin{sphinxuseclass}{cell}\begin{sphinxVerbatimInput}

\begin{sphinxuseclass}{cell_input}
\begin{sphinxVerbatim}[commandchars=\\\{\}]
\PYG{k+kn}{from} \PYG{n+nn}{myst\PYGZus{}nb} \PYG{k+kn}{import} \PYG{n}{glue}
\end{sphinxVerbatim}

\end{sphinxuseclass}\end{sphinxVerbatimInput}

\end{sphinxuseclass}
\sphinxAtStartPar
Veamos un ejemplo de como usarlo:

\begin{sphinxuseclass}{cell}\begin{sphinxVerbatimInput}

\begin{sphinxuseclass}{cell_input}
\begin{sphinxVerbatim}[commandchars=\\\{\}]
\PYG{n}{my\PYGZus{}variable} \PYG{o}{=} \PYG{l+s+s2}{\PYGZdq{}}\PYG{l+s+s2}{here is some text!}\PYG{l+s+s2}{\PYGZdq{}}
\PYG{n}{glue}\PYG{p}{(}\PYG{l+s+s2}{\PYGZdq{}}\PYG{l+s+s2}{cool\PYGZus{}text}\PYG{l+s+s2}{\PYGZdq{}}\PYG{p}{,} \PYG{n}{my\PYGZus{}variable}\PYG{p}{,}  \PYG{n}{display}\PYG{o}{=}\PYG{k+kc}{False}\PYG{p}{)}
\end{sphinxVerbatim}

\end{sphinxuseclass}\end{sphinxVerbatimInput}

\end{sphinxuseclass}
\sphinxAtStartPar
Para llamarla usamos \sphinxcode{\sphinxupquote{\{glue:\}`cool\_text`}}: \DUrole{output,text_plain}{‘here is some text!’}


\subsubsection{“Gluing” numeros, plots. math y tablas}
\label{\detokenize{docs/01_01_Code_Blocks:gluing-numeros-plots-math-y-tablas}}
\begin{sphinxuseclass}{cell}\begin{sphinxVerbatimInput}

\begin{sphinxuseclass}{cell_input}
\begin{sphinxVerbatim}[commandchars=\\\{\}]
\PYG{c+c1}{\PYGZsh{} Simulate some data and bootstrap the mean of the data}
\PYG{k+kn}{import} \PYG{n+nn}{numpy} \PYG{k}{as} \PYG{n+nn}{np}
\PYG{k+kn}{import} \PYG{n+nn}{pandas} \PYG{k}{as} \PYG{n+nn}{pd}
\PYG{k+kn}{import} \PYG{n+nn}{matplotlib}\PYG{n+nn}{.}\PYG{n+nn}{pyplot} \PYG{k}{as} \PYG{n+nn}{plt}

\PYG{n}{n\PYGZus{}points} \PYG{o}{=} \PYG{l+m+mi}{10000}
\PYG{n}{n\PYGZus{}boots} \PYG{o}{=} \PYG{l+m+mi}{1000}
\PYG{n}{mean}\PYG{p}{,} \PYG{n}{sd} \PYG{o}{=} \PYG{p}{(}\PYG{l+m+mi}{3}\PYG{p}{,} \PYG{l+m+mf}{.2}\PYG{p}{)}
\PYG{n}{data} \PYG{o}{=} \PYG{n}{sd}\PYG{o}{*}\PYG{n}{np}\PYG{o}{.}\PYG{n}{random}\PYG{o}{.}\PYG{n}{randn}\PYG{p}{(}\PYG{n}{n\PYGZus{}points}\PYG{p}{)} \PYG{o}{+} \PYG{n}{mean}
\PYG{n}{bootstrap\PYGZus{}indices} \PYG{o}{=} \PYG{n}{np}\PYG{o}{.}\PYG{n}{random}\PYG{o}{.}\PYG{n}{randint}\PYG{p}{(}\PYG{l+m+mi}{0}\PYG{p}{,} \PYG{n}{n\PYGZus{}points}\PYG{p}{,} \PYG{n}{n\PYGZus{}points}\PYG{o}{*}\PYG{n}{n\PYGZus{}boots}\PYG{p}{)}\PYG{o}{.}\PYG{n}{reshape}\PYG{p}{(}\PYG{p}{(}\PYG{n}{n\PYGZus{}boots}\PYG{p}{,} \PYG{n}{n\PYGZus{}points}\PYG{p}{)}\PYG{p}{)}

\PYG{c+c1}{\PYGZsh{} Calculate the mean of a bunch of random samples}
\PYG{n}{means} \PYG{o}{=} \PYG{n}{data}\PYG{p}{[}\PYG{n}{bootstrap\PYGZus{}indices}\PYG{p}{]}\PYG{o}{.}\PYG{n}{mean}\PYG{p}{(}\PYG{l+m+mi}{0}\PYG{p}{)}
\PYG{c+c1}{\PYGZsh{} Calculate the 95\PYGZpc{} confidence interval for the mean}
\PYG{n}{clo}\PYG{p}{,} \PYG{n}{chi} \PYG{o}{=} \PYG{n}{np}\PYG{o}{.}\PYG{n}{percentile}\PYG{p}{(}\PYG{n}{means}\PYG{p}{,} \PYG{p}{[}\PYG{l+m+mf}{2.5}\PYG{p}{,} \PYG{l+m+mf}{97.5}\PYG{p}{]}\PYG{p}{)}

\PYG{c+c1}{\PYGZsh{} Visualize the histogram with the intervals}
\PYG{n}{fig}\PYG{p}{,} \PYG{n}{ax} \PYG{o}{=} \PYG{n}{plt}\PYG{o}{.}\PYG{n}{subplots}\PYG{p}{(}\PYG{p}{)}
\PYG{n}{ax}\PYG{o}{.}\PYG{n}{hist}\PYG{p}{(}\PYG{n}{means}\PYG{p}{)}
\PYG{k}{for} \PYG{n}{ln} \PYG{o+ow}{in} \PYG{p}{[}\PYG{n}{clo}\PYG{p}{,} \PYG{n}{chi}\PYG{p}{]}\PYG{p}{:}
    \PYG{n}{ax}\PYG{o}{.}\PYG{n}{axvline}\PYG{p}{(}\PYG{n}{ln}\PYG{p}{,} \PYG{n}{ls}\PYG{o}{=}\PYG{l+s+s1}{\PYGZsq{}}\PYG{l+s+s1}{\PYGZhy{}\PYGZhy{}}\PYG{l+s+s1}{\PYGZsq{}}\PYG{p}{,} \PYG{n}{c}\PYG{o}{=}\PYG{l+s+s1}{\PYGZsq{}}\PYG{l+s+s1}{r}\PYG{l+s+s1}{\PYGZsq{}}\PYG{p}{)}
\PYG{n}{ax}\PYG{o}{.}\PYG{n}{set\PYGZus{}title}\PYG{p}{(}\PYG{l+s+s2}{\PYGZdq{}}\PYG{l+s+s2}{Bootstrap distribution and 95}\PYG{l+s+s2}{\PYGZpc{}}\PYG{l+s+s2}{ CI}\PYG{l+s+s2}{\PYGZdq{}}\PYG{p}{)}

\PYG{c+c1}{\PYGZsh{} And a wider figure to show a timeseries}
\PYG{n}{fig2}\PYG{p}{,} \PYG{n}{ax} \PYG{o}{=} \PYG{n}{plt}\PYG{o}{.}\PYG{n}{subplots}\PYG{p}{(}\PYG{n}{figsize}\PYG{o}{=}\PYG{p}{(}\PYG{l+m+mi}{6}\PYG{p}{,} \PYG{l+m+mi}{2}\PYG{p}{)}\PYG{p}{)}
\PYG{n}{ax}\PYG{o}{.}\PYG{n}{plot}\PYG{p}{(}\PYG{n}{np}\PYG{o}{.}\PYG{n}{sort}\PYG{p}{(}\PYG{n}{means}\PYG{p}{)}\PYG{p}{,} \PYG{n}{lw}\PYG{o}{=}\PYG{l+m+mi}{3}\PYG{p}{,} \PYG{n}{c}\PYG{o}{=}\PYG{l+s+s1}{\PYGZsq{}}\PYG{l+s+s1}{r}\PYG{l+s+s1}{\PYGZsq{}}\PYG{p}{)}
\PYG{n}{ax}\PYG{o}{.}\PYG{n}{set\PYGZus{}axis\PYGZus{}off}\PYG{p}{(}\PYG{p}{)}


\PYG{c+c1}{\PYGZsh{} Store the values in our notebook}

\PYG{n}{glue}\PYG{p}{(}\PYG{l+s+s2}{\PYGZdq{}}\PYG{l+s+s2}{boot\PYGZus{}mean}\PYG{l+s+s2}{\PYGZdq{}}\PYG{p}{,} \PYG{n}{means}\PYG{o}{.}\PYG{n}{mean}\PYG{p}{(}\PYG{p}{)}\PYG{p}{,} \PYG{n}{display}\PYG{o}{=}\PYG{k+kc}{False}\PYG{p}{)} \PYG{c+c1}{\PYGZsh{} numero}
\PYG{n}{glue}\PYG{p}{(}\PYG{l+s+s2}{\PYGZdq{}}\PYG{l+s+s2}{boot\PYGZus{}clo}\PYG{l+s+s2}{\PYGZdq{}}\PYG{p}{,} \PYG{n}{clo}\PYG{p}{,} \PYG{n}{display}\PYG{o}{=}\PYG{k+kc}{False}\PYG{p}{)}           \PYG{c+c1}{\PYGZsh{} numero}
\PYG{n}{glue}\PYG{p}{(}\PYG{l+s+s2}{\PYGZdq{}}\PYG{l+s+s2}{boot\PYGZus{}chi}\PYG{l+s+s2}{\PYGZdq{}}\PYG{p}{,} \PYG{n}{chi}\PYG{p}{,} \PYG{n}{display}\PYG{o}{=}\PYG{k+kc}{False}\PYG{p}{)}           \PYG{c+c1}{\PYGZsh{} numero}

\PYG{n}{glue}\PYG{p}{(}\PYG{l+s+s2}{\PYGZdq{}}\PYG{l+s+s2}{boot\PYGZus{}fig}\PYG{l+s+s2}{\PYGZdq{}}\PYG{p}{,} \PYG{n}{fig}\PYG{p}{,} \PYG{n}{display}\PYG{o}{=}\PYG{k+kc}{False}\PYG{p}{)}           \PYG{c+c1}{\PYGZsh{} Plot}
\PYG{n}{glue}\PYG{p}{(}\PYG{l+s+s2}{\PYGZdq{}}\PYG{l+s+s2}{sorted\PYGZus{}means\PYGZus{}fig}\PYG{l+s+s2}{\PYGZdq{}}\PYG{p}{,} \PYG{n}{fig2}\PYG{p}{,} \PYG{n}{display}\PYG{o}{=}\PYG{k+kc}{False}\PYG{p}{)}  \PYG{c+c1}{\PYGZsh{} Plot}

\PYG{c+c1}{\PYGZsh{} Dataframes}

\PYG{n}{bootstrap\PYGZus{}subsets} \PYG{o}{=} \PYG{n}{data}\PYG{p}{[}\PYG{n}{bootstrap\PYGZus{}indices}\PYG{p}{]}\PYG{p}{[}\PYG{p}{:}\PYG{l+m+mi}{3}\PYG{p}{,} \PYG{p}{:}\PYG{l+m+mi}{5}\PYG{p}{]}\PYG{o}{.}\PYG{n}{T}
\PYG{n}{df} \PYG{o}{=} \PYG{n}{pd}\PYG{o}{.}\PYG{n}{DataFrame}\PYG{p}{(}\PYG{n}{bootstrap\PYGZus{}subsets}\PYG{p}{,} \PYG{n}{columns}\PYG{o}{=}\PYG{p}{[}\PYG{l+s+s2}{\PYGZdq{}}\PYG{l+s+s2}{first}\PYG{l+s+s2}{\PYGZdq{}}\PYG{p}{,} \PYG{l+s+s2}{\PYGZdq{}}\PYG{l+s+s2}{second}\PYG{l+s+s2}{\PYGZdq{}}\PYG{p}{,} \PYG{l+s+s2}{\PYGZdq{}}\PYG{l+s+s2}{third}\PYG{l+s+s2}{\PYGZdq{}}\PYG{p}{]}\PYG{p}{)}
\PYG{n}{display}\PYG{p}{(}\PYG{n}{df}\PYG{p}{)}

\PYG{n}{glue}\PYG{p}{(}\PYG{l+s+s2}{\PYGZdq{}}\PYG{l+s+s2}{df\PYGZus{}tbl}\PYG{l+s+s2}{\PYGZdq{}}\PYG{p}{,} \PYG{n}{df}\PYG{p}{,} \PYG{n}{display}\PYG{o}{=}\PYG{k+kc}{False}\PYG{p}{)}
\end{sphinxVerbatim}

\end{sphinxuseclass}\end{sphinxVerbatimInput}
\begin{sphinxVerbatimOutput}

\begin{sphinxuseclass}{cell_output}
\begin{sphinxVerbatim}[commandchars=\\\{\}]
      first    second     third
0  2.864279  3.096170  3.040294
1  2.850520  3.582923  2.632284
2  3.182612  3.026727  3.013184
3  3.159771  2.793257  3.151598
4  3.087032  3.159038  3.132074
\end{sphinxVerbatim}

\noindent\sphinxincludegraphics{{374c4f574ad7f704345e02f565e8d3e68cfc5199e0048c646a06fdce10717658}.png}

\noindent\sphinxincludegraphics{{63154edfac87dcf5eed7313effd8018b3e42af4c28e07b22ea99ae3318bde508}.png}

\end{sphinxuseclass}\end{sphinxVerbatimOutput}

\end{sphinxuseclass}

\subsubsection{“Pasting” las variables}
\label{\detokenize{docs/01_01_Code_Blocks:pasting-las-variables}}

\paragraph{glue:any (sin formato)}
\label{\detokenize{docs/01_01_Code_Blocks:glue-any-sin-formato}}
\sphinxAtStartPar
Por defecto, al usar \sphinxcode{\sphinxupquote{\{glue:\}}} estamos usando \sphinxcode{\sphinxupquote{\{glue:any\}}}, que pega la salida
“encolada” en línea o como bloque, respectivamente, sin formato adicional.

\sphinxAtStartPar
Veamos un ejemplo:

\begin{sphinxVerbatim}[commandchars=\\\{\}]
In\PYGZhy{}line text; \PYGZob{}glue:\PYGZcb{}`boot\PYGZus{}mean`, and a figure: \PYGZob{}glue:\PYGZcb{}`boot\PYGZus{}fig`.
\end{sphinxVerbatim}

\sphinxAtStartPar
In\sphinxhyphen{}line text; \DUrole{output,text_plain}{2.99758724978736}, and a figure: \sphinxincludegraphics{{374c4f574ad7f704345e02f565e8d3e68cfc5199e0048c646a06fdce10717658}.png}.


\paragraph{glue:text}
\label{\detokenize{docs/01_01_Code_Blocks:glue-text}}
\sphinxAtStartPar
El \sphinxcode{\sphinxupquote{glue:text}} es específico àra textos planos. Veamos un ejemplo:
\begin{quote}

\begin{sphinxVerbatim}[commandchars=\\\{\}]
The mean of the bootstrapped distribution was \PYGZob{}glue:text\PYGZcb{}`boot\PYGZus{}mean` (95\PYGZpc{} confidence interval \PYGZob{}glue:text\PYGZcb{}`boot\PYGZus{}clo`/\PYGZob{}glue:text\PYGZcb{}`boot\PYGZus{}chi`).
\end{sphinxVerbatim}

\sphinxAtStartPar
The mean of the bootstrapped distribution was \DUrole{pasted-text}{2.99758724978736} (95\% confidence interval \DUrole{pasted-text}{2.985312582852057}/\DUrole{pasted-text}{3.0098125309029817}).
\end{quote}

\sphinxAtStartPar
Podemos darle formato al output, como redondear números. La sintaxis es
\begin{itemize}
\item {} 
\sphinxAtStartPar
\sphinxcode{\sphinxupquote{\{glue:text\}`mykey:formatstring`}}

\end{itemize}

\sphinxAtStartPar
Por ejemplo:
\begin{quote}

\begin{sphinxVerbatim}[commandchars=\\\{\}]
Media: \PYGZob{}glue:text\PYGZcb{}`boot\PYGZus{}mean``

Media redondeada: \PYGZob{}glue:text\PYGZcb{}`boot\PYGZus{}mean:.2f`
\end{sphinxVerbatim}

\sphinxAtStartPar
Media: \DUrole{pasted-text}{2.99758724978736}

\sphinxAtStartPar
Media redondeada: \DUrole{pasted-text}{3.00}
\end{quote}


\paragraph{glue:figure}
\label{\detokenize{docs/01_01_Code_Blocks:glue-figure}}
\sphinxAtStartPar
Sirve para figuras y tablas (dataframes).

\sphinxAtStartPar
Figura:
\begin{quote}

\begin{sphinxVerbatim}[commandchars=\\\{\}]
```\PYGZob{}glue:figure\PYGZcb{} boot\PYGZus{}fig
:figwidth: 300px
:name: \PYGZdq{}fig\PYGZhy{}boot\PYGZdq{}

This is a **caption**, with an embedded `\PYGZob{}glue:text\PYGZcb{}` element: \PYGZob{}glue:text\PYGZcb{}`boot\PYGZus{}mean:.2f`!
```
\end{sphinxVerbatim}

\begin{figure}[htbp]
\centering
\capstart

\noindent\sphinxincludegraphics{{374c4f574ad7f704345e02f565e8d3e68cfc5199e0048c646a06fdce10717658}.png}
\caption{This is a \sphinxstylestrong{caption}, with an embedded \sphinxcode{\sphinxupquote{\{glue:text\}}} element: \DUrole{pasted-text}{3.00}!}\label{\detokenize{docs/01_01_Code_Blocks:fig-boot}}\end{figure}

\begin{sphinxVerbatim}[commandchars=\\\{\}]
Here is a \PYGZob{}ref\PYGZcb{}`reference to the figure \PYGZlt{}fig\PYGZhy{}boot\PYGZgt{}`
\end{sphinxVerbatim}

\sphinxAtStartPar
Here is a {\hyperref[\detokenize{docs/01_01_Code_Blocks:fig-boot}]{\sphinxcrossref{\DUrole{std,std-ref}{reference to the figure}}}}
\end{quote}

\sphinxAtStartPar
Dataframe:
\begin{quote}

\begin{sphinxVerbatim}[commandchars=\\\{\}]
```\PYGZob{}glue:figure\PYGZcb{} df\PYGZus{}tbl
:figwidth: 300px
:name: \PYGZdq{}tbl:df\PYGZdq{}

A caption for a pandas table.
```
\end{sphinxVerbatim}

\begin{figure}[htbp]
\centering
\capstart

\begin{sphinxVerbatim}[commandchars=\\\{\}]
      first    second     third
0  2.864279  3.096170  3.040294
1  2.850520  3.582923  2.632284
2  3.182612  3.026727  3.013184
3  3.159771  2.793257  3.151598
4  3.087032  3.159038  3.132074
\end{sphinxVerbatim}
\caption{A caption for a pandas table.}\label{\detokenize{docs/01_01_Code_Blocks:tbl-df}}\end{figure}
\end{quote}


\paragraph{glue:math}
\label{\detokenize{docs/01_01_Code_Blocks:glue-math}}
\begin{sphinxuseclass}{cell}\begin{sphinxVerbatimInput}

\begin{sphinxuseclass}{cell_input}
\begin{sphinxVerbatim}[commandchars=\\\{\}]
\PYG{k+kn}{import} \PYG{n+nn}{sympy} \PYG{k}{as} \PYG{n+nn}{sym}
\PYG{n}{f} \PYG{o}{=} \PYG{n}{sym}\PYG{o}{.}\PYG{n}{Function}\PYG{p}{(}\PYG{l+s+s1}{\PYGZsq{}}\PYG{l+s+s1}{f}\PYG{l+s+s1}{\PYGZsq{}}\PYG{p}{)}
\PYG{n}{y} \PYG{o}{=} \PYG{n}{sym}\PYG{o}{.}\PYG{n}{Function}\PYG{p}{(}\PYG{l+s+s1}{\PYGZsq{}}\PYG{l+s+s1}{y}\PYG{l+s+s1}{\PYGZsq{}}\PYG{p}{)}
\PYG{n}{n} \PYG{o}{=} \PYG{n}{sym}\PYG{o}{.}\PYG{n}{symbols}\PYG{p}{(}\PYG{l+s+sa}{r}\PYG{l+s+s1}{\PYGZsq{}}\PYG{l+s+s1}{\PYGZbs{}}\PYG{l+s+s1}{alpha}\PYG{l+s+s1}{\PYGZsq{}}\PYG{p}{)}
\PYG{n}{f} \PYG{o}{=} \PYG{n}{y}\PYG{p}{(}\PYG{n}{n}\PYG{p}{)}\PYG{o}{\PYGZhy{}}\PYG{l+m+mi}{2}\PYG{o}{*}\PYG{n}{y}\PYG{p}{(}\PYG{n}{n}\PYG{o}{\PYGZhy{}}\PYG{l+m+mi}{1}\PYG{o}{/}\PYG{n}{sym}\PYG{o}{.}\PYG{n}{pi}\PYG{p}{)}\PYG{o}{\PYGZhy{}}\PYG{l+m+mi}{5}\PYG{o}{*}\PYG{n}{y}\PYG{p}{(}\PYG{n}{n}\PYG{o}{\PYGZhy{}}\PYG{l+m+mi}{2}\PYG{p}{)}
\PYG{n}{glue}\PYG{p}{(}\PYG{l+s+s2}{\PYGZdq{}}\PYG{l+s+s2}{sym\PYGZus{}eq}\PYG{l+s+s2}{\PYGZdq{}}\PYG{p}{,} \PYG{n}{sym}\PYG{o}{.}\PYG{n}{rsolve}\PYG{p}{(}\PYG{n}{f}\PYG{p}{,}\PYG{n}{y}\PYG{p}{(}\PYG{n}{n}\PYG{p}{)}\PYG{p}{,}\PYG{p}{[}\PYG{l+m+mi}{1}\PYG{p}{,}\PYG{l+m+mi}{4}\PYG{p}{]}\PYG{p}{)} \PYG{p}{,}\PYG{n}{display}\PYG{o}{=}\PYG{k+kc}{False}\PYG{p}{)}
\end{sphinxVerbatim}

\end{sphinxuseclass}\end{sphinxVerbatimInput}

\end{sphinxuseclass}\begin{quote}

\begin{sphinxVerbatim}[commandchars=\\\{\}]
```\PYGZob{}glue:math\PYGZcb{} sym\PYGZus{}eq
:label: eq\PYGZhy{}sym
``
\end{sphinxVerbatim}
\begin{equation}\label{equation:docs/01_01_Code_Blocks:eq-sym}
\begin{split}\displaystyle \left(\sqrt{5} i\right)^{\alpha} \left(\frac{1}{2} - \frac{2 \sqrt{5} i}{5}\right) + \left(- \sqrt{5} i\right)^{\alpha} \left(\frac{1}{2} + \frac{2 \sqrt{5} i}{5}\right)\end{split}
\end{equation}\end{quote}


\subsubsection{“Pasting” en tablas}
\label{\detokenize{docs/01_01_Code_Blocks:pasting-en-tablas}}\begin{quote}

\begin{sphinxVerbatim}[commandchars=\\\{\}]
| name                            |       plot                    | mean                      | ci                                                |
|:\PYGZhy{}\PYGZhy{}\PYGZhy{}\PYGZhy{}\PYGZhy{}\PYGZhy{}\PYGZhy{}\PYGZhy{}\PYGZhy{}\PYGZhy{}\PYGZhy{}\PYGZhy{}\PYGZhy{}\PYGZhy{}\PYGZhy{}\PYGZhy{}\PYGZhy{}\PYGZhy{}\PYGZhy{}\PYGZhy{}\PYGZhy{}\PYGZhy{}\PYGZhy{}\PYGZhy{}\PYGZhy{}\PYGZhy{}\PYGZhy{}\PYGZhy{}\PYGZhy{}\PYGZhy{}\PYGZhy{}:|:\PYGZhy{}\PYGZhy{}\PYGZhy{}\PYGZhy{}\PYGZhy{}\PYGZhy{}\PYGZhy{}\PYGZhy{}\PYGZhy{}\PYGZhy{}\PYGZhy{}\PYGZhy{}\PYGZhy{}\PYGZhy{}\PYGZhy{}\PYGZhy{}\PYGZhy{}\PYGZhy{}\PYGZhy{}\PYGZhy{}\PYGZhy{}\PYGZhy{}\PYGZhy{}\PYGZhy{}\PYGZhy{}\PYGZhy{}\PYGZhy{}\PYGZhy{}\PYGZhy{}:|\PYGZhy{}\PYGZhy{}\PYGZhy{}\PYGZhy{}\PYGZhy{}\PYGZhy{}\PYGZhy{}\PYGZhy{}\PYGZhy{}\PYGZhy{}\PYGZhy{}\PYGZhy{}\PYGZhy{}\PYGZhy{}\PYGZhy{}\PYGZhy{}\PYGZhy{}\PYGZhy{}\PYGZhy{}\PYGZhy{}\PYGZhy{}\PYGZhy{}\PYGZhy{}\PYGZhy{}\PYGZhy{}\PYGZhy{}\PYGZhy{}|\PYGZhy{}\PYGZhy{}\PYGZhy{}\PYGZhy{}\PYGZhy{}\PYGZhy{}\PYGZhy{}\PYGZhy{}\PYGZhy{}\PYGZhy{}\PYGZhy{}\PYGZhy{}\PYGZhy{}\PYGZhy{}\PYGZhy{}\PYGZhy{}\PYGZhy{}\PYGZhy{}\PYGZhy{}\PYGZhy{}\PYGZhy{}\PYGZhy{}\PYGZhy{}\PYGZhy{}\PYGZhy{}\PYGZhy{}\PYGZhy{}\PYGZhy{}\PYGZhy{}\PYGZhy{}\PYGZhy{}\PYGZhy{}\PYGZhy{}\PYGZhy{}\PYGZhy{}\PYGZhy{}\PYGZhy{}\PYGZhy{}\PYGZhy{}\PYGZhy{}\PYGZhy{}\PYGZhy{}\PYGZhy{}\PYGZhy{}\PYGZhy{}\PYGZhy{}\PYGZhy{}\PYGZhy{}\PYGZhy{}\PYGZhy{}\PYGZhy{}|
| histogram and raw text          | \PYGZob{}glue:\PYGZcb{}`boot\PYGZus{}fig`             | \PYGZob{}glue:\PYGZcb{}`boot\PYGZus{}mean`          | \PYGZob{}glue:\PYGZcb{}`boot\PYGZus{}clo`\PYGZhy{}\PYGZob{}glue:\PYGZcb{}`boot\PYGZus{}chi`                   |
| sorted means and formatted text | \PYGZob{}glue:\PYGZcb{}`sorted\PYGZus{}means\PYGZus{}fig`     | \PYGZob{}glue:text\PYGZcb{}`boot\PYGZus{}mean:.3f` | \PYGZob{}glue:text\PYGZcb{}`boot\PYGZus{}clo:.3f`\PYGZhy{}\PYGZob{}glue:text\PYGZcb{}`boot\PYGZus{}chi:.3f` |
\end{sphinxVerbatim}


\begin{savenotes}\sphinxattablestart
\centering
\begin{tabulary}{\linewidth}[t]{|T|T|T|T|}
\hline
\sphinxstyletheadfamily 
\sphinxAtStartPar
name
&\sphinxstyletheadfamily 
\sphinxAtStartPar
plot
&\sphinxstyletheadfamily 
\sphinxAtStartPar
mean
&\sphinxstyletheadfamily 
\sphinxAtStartPar
ci
\\
\hline
\sphinxAtStartPar
histogram and raw text
&
\sphinxAtStartPar
\sphinxincludegraphics{{374c4f574ad7f704345e02f565e8d3e68cfc5199e0048c646a06fdce10717658}.png}
&
\sphinxAtStartPar
\DUrole{output,text_plain}{2.99758724978736}
&
\sphinxAtStartPar
\DUrole{output,text_plain}{2.985312582852057}\sphinxhyphen{}\DUrole{output,text_plain}{3.0098125309029817}
\\
\hline
\sphinxAtStartPar
sorted means and formatted text
&
\sphinxAtStartPar
\sphinxincludegraphics{{63154edfac87dcf5eed7313effd8018b3e42af4c28e07b22ea99ae3318bde508}.png}
&
\sphinxAtStartPar
\DUrole{pasted-text}{2.998}
&
\sphinxAtStartPar
\DUrole{pasted-text}{2.985}\sphinxhyphen{}\DUrole{pasted-text}{3.010}
\\
\hline
\end{tabulary}
\par
\sphinxattableend\end{savenotes}
\end{quote}


\subsection{Estadisticas de las ejecuciones}
\label{\detokenize{docs/01_01_Code_Blocks:estadisticas-de-las-ejecuciones}}

\begin{savenotes}\sphinxattablestart
\centering
\begin{tabulary}{\linewidth}[t]{|T|T|T|T|T|}
\hline
\sphinxstyletheadfamily 
\sphinxAtStartPar
Document
&\sphinxstyletheadfamily 
\sphinxAtStartPar
Modified
&\sphinxstyletheadfamily 
\sphinxAtStartPar
Method
&\sphinxstyletheadfamily 
\sphinxAtStartPar
Run Time (s)
&\sphinxstyletheadfamily 
\sphinxAtStartPar
Status
\\
\hline
\sphinxAtStartPar
{\hyperref[\detokenize{docs/01_00_Code_Blocks_y_ecuaciones::doc}]{\sphinxcrossref{\DUrole{doc}{docs/01\_00\_Code\_Blocks\_y\_ecuaciones}}}}
&
\sphinxAtStartPar
2023\sphinxhyphen{}11\sphinxhyphen{}30 10:00
&
\sphinxAtStartPar
cache
&
\sphinxAtStartPar
0.68
&
\sphinxAtStartPar
✅
\\
\hline
\sphinxAtStartPar
{\hyperref[\detokenize{docs/01_01_Code_Blocks::doc}]{\sphinxcrossref{\DUrole{doc}{docs/01\_01\_Code\_Blocks}}}}
&
\sphinxAtStartPar
2023\sphinxhyphen{}12\sphinxhyphen{}11 09:16
&
\sphinxAtStartPar
cache
&
\sphinxAtStartPar
2.44
&
\sphinxAtStartPar
✅
\\
\hline
\sphinxAtStartPar
{\hyperref[\detokenize{docs/01_02_Ecuaciones::doc}]{\sphinxcrossref{\DUrole{doc}{docs/01\_02\_Ecuaciones}}}}
&
\sphinxAtStartPar
2023\sphinxhyphen{}11\sphinxhyphen{}30 10:00
&
\sphinxAtStartPar
cache
&
\sphinxAtStartPar
0.68
&
\sphinxAtStartPar
✅
\\
\hline
\sphinxAtStartPar
{\hyperref[\detokenize{docs/02_00_Roles_and_directives::doc}]{\sphinxcrossref{\DUrole{doc}{docs/02\_00\_Roles\_and\_directives}}}}
&
\sphinxAtStartPar
2023\sphinxhyphen{}11\sphinxhyphen{}30 10:00
&
\sphinxAtStartPar
cache
&
\sphinxAtStartPar
0.68
&
\sphinxAtStartPar
✅
\\
\hline
\sphinxAtStartPar
{\hyperref[\detokenize{docs/02_01_Cuadros::doc}]{\sphinxcrossref{\DUrole{doc}{docs/02\_01\_Cuadros}}}}
&
\sphinxAtStartPar
2023\sphinxhyphen{}11\sphinxhyphen{}30 10:00
&
\sphinxAtStartPar
cache
&
\sphinxAtStartPar
0.68
&
\sphinxAtStartPar
✅
\\
\hline
\sphinxAtStartPar
{\hyperref[\detokenize{docs/02_02_Footnotes::doc}]{\sphinxcrossref{\DUrole{doc}{docs/02\_02\_Footnotes}}}}
&
\sphinxAtStartPar
2023\sphinxhyphen{}11\sphinxhyphen{}30 10:00
&
\sphinxAtStartPar
cache
&
\sphinxAtStartPar
0.68
&
\sphinxAtStartPar
✅
\\
\hline
\sphinxAtStartPar
{\hyperref[\detokenize{docs/02_03_Figuras::doc}]{\sphinxcrossref{\DUrole{doc}{docs/02\_03\_Figuras}}}}
&
\sphinxAtStartPar
2023\sphinxhyphen{}11\sphinxhyphen{}30 10:00
&
\sphinxAtStartPar
cache
&
\sphinxAtStartPar
0.68
&
\sphinxAtStartPar
✅
\\
\hline
\sphinxAtStartPar
{\hyperref[\detokenize{docs/02_04_Mas_cosas::doc}]{\sphinxcrossref{\DUrole{doc}{docs/02\_04\_Mas\_cosas}}}}
&
\sphinxAtStartPar
2023\sphinxhyphen{}11\sphinxhyphen{}30 10:00
&
\sphinxAtStartPar
cache
&
\sphinxAtStartPar
0.68
&
\sphinxAtStartPar
✅
\\
\hline
\sphinxAtStartPar
{\hyperref[\detokenize{docs/03_00_referenciar_cosas::doc}]{\sphinxcrossref{\DUrole{doc}{docs/03\_00\_referenciar\_cosas}}}}
&
\sphinxAtStartPar
2023\sphinxhyphen{}11\sphinxhyphen{}30 10:00
&
\sphinxAtStartPar
cache
&
\sphinxAtStartPar
0.68
&
\sphinxAtStartPar
✅
\\
\hline
\sphinxAtStartPar
{\hyperref[\detokenize{docs/04_00_Teoremas_pruebas_y_algoritmos::doc}]{\sphinxcrossref{\DUrole{doc}{docs/04\_00\_Teoremas\_pruebas\_y\_algoritmos}}}}
&
\sphinxAtStartPar
2023\sphinxhyphen{}11\sphinxhyphen{}30 10:00
&
\sphinxAtStartPar
cache
&
\sphinxAtStartPar
0.68
&
\sphinxAtStartPar
✅
\\
\hline
\end{tabulary}
\par
\sphinxattableend\end{savenotes}

\sphinxstepscope
\begin{quote}

\sphinxAtStartPar
Dec 11, 2023 | 25 words | 0 min read
\end{quote}


\section{Ecuaciones}
\label{\detokenize{docs/01_02_Ecuaciones:ecuaciones}}\label{\detokenize{docs/01_02_Ecuaciones::doc}}\label{equation:docs/01_02_Ecuaciones:677f3939-8a33-420e-bf8b-a9355753849c}\begin{equation} 
    f(x) = x^2 
\end{equation}\begin{equation}\label{equation:docs/01_02_Ecuaciones:my_other_label}
\begin{split}
    w_{t+1} = (1 + r_{t+1}) s(w_t) + y_{t+1}
\end{split}
\end{equation}
\sphinxAtStartPar
A link to a dollar math block: \eqref{equation:docs/01_02_Ecuaciones:my_other_label}
\begin{equation}\label{equation:docs/01_02_Ecuaciones:euler}
\begin{split}        e^{i\pi} + 1 = 0\end{split}
\end{equation}
\sphinxAtStartPar
Refereciemos la ec. de euler \eqref{equation:docs/01_02_Ecuaciones:euler}.
\begin{equation}\label{equation:docs/01_02_Ecuaciones:label_align}
\begin{split}        \begin{align}
        y    & =  ax^2 + bx + c \\
        f(x) & =  x^2 + 2xy + y^2 
        \end{align}\end{split}
\end{equation}
\sphinxAtStartPar
Refereciemos la ec. \eqref{equation:docs/01_02_Ecuaciones:label_align}.
\begin{equation}\label{equation:docs/01_02_Ecuaciones:label_align_dolar}
\begin{split}
    \begin{aligned}
    y    & =  ax^2 + bx + c \\
    f(x) & =  x^2 + 2xy + y^2 
    \end{aligned}
\end{split}
\end{equation}
\sphinxAtStartPar
Refereciemos la ec. \eqref{equation:docs/01_02_Ecuaciones:label_align_dolar}.
\begin{equation}\label{equation:docs/01_02_Ecuaciones:label_1}
\begin{split}        (a + b)^2  &=  (a + b)(a + b) \\
                   &=  a^2 + 2ab + b^2\end{split}
\end{equation}
\sphinxAtStartPar
Refereciemos la ec. \eqref{equation:docs/01_02_Ecuaciones:label_1}.

\sphinxstepscope
\begin{quote}

\sphinxAtStartPar
Dec 11, 2023 | 22 words | 0 min read
\end{quote}


\chapter{Roles and directives}
\label{\detokenize{docs/02_00_Roles_and_directives:roles-and-directives}}\label{\detokenize{docs/02_00_Roles_and_directives::doc}}
\sphinxAtStartPar
Basicamente son funciones. \sphinxurl{https://myst-parser.readthedocs.io/en/latest/syntax/roles-and-directives.html\#syntax-directives}

\sphinxAtStartPar
Las \sphinxstylestrong{directives} son funciones de varias lineas. Los \sphinxstylestrong{roles} son de una línea.

\sphinxstepscope
\begin{quote}

\sphinxAtStartPar
Dec 11, 2023 | 284 words | 1 min read
\end{quote}


\section{Cuadros (admonitions)}
\label{\detokenize{docs/02_01_Cuadros:cuadros-admonitions}}\label{\detokenize{docs/02_01_Cuadros::doc}}

\subsection{Cuadros donde se puede cambiar el título}
\label{\detokenize{docs/02_01_Cuadros:cuadros-donde-se-puede-cambiar-el-titulo}}
\begin{sphinxadmonition}{note}{This is an admonition}

\sphinxAtStartPar
This is an admonition
\end{sphinxadmonition}

\begin{sphinxadmonition}{note}{This is an admonition class note}

\sphinxAtStartPar
This is an admonition class note
\end{sphinxadmonition}

\begin{sphinxadmonition}{note}{This is an admonition class important}

\sphinxAtStartPar
This is an admonition class important
\end{sphinxadmonition}

\begin{sphinxadmonition}{note}{This is an admonition class warning}

\sphinxAtStartPar
This is an admonition class warning
\end{sphinxadmonition}

\begin{sphinxadmonition}{note}{This is an admonition class tip}

\sphinxAtStartPar
This is an admonition class tip
\end{sphinxadmonition}

\begin{sphinxadmonition}{note}{This is an admonition class tip}

\sphinxAtStartPar
This is an admonition class tip
\end{sphinxadmonition}

\begin{sphinxVerbatim}[commandchars=\\\{\}]
Esto es para escribir texto plano. Es decir, no se renderizan 
los comandos estilo \PYGZob{}numref\PYGZcb{}`Figure \PYGZpc{}s \PYGZlt{}fig\PYGZhy{}target\PYGZus{}3\PYGZgt{}`
\end{sphinxVerbatim}
\begin{quote}

\sphinxAtStartPar
Quote block

\sphinxAtStartPar
Y vemos que puede ser de varias líneas
\end{quote}


\subsection{Cuadros rápidos (no se puede cambiar el título)}
\label{\detokenize{docs/02_01_Cuadros:cuadros-rapidos-no-se-puede-cambiar-el-titulo}}
\begin{sphinxadmonition}{note}{Note:}
\sphinxAtStartPar
This a note (no se puede cambiar el titulo)
\end{sphinxadmonition}

\begin{sphinxadmonition}{warning}{Warning:}
\sphinxAtStartPar
This is a warning (no se puede cambiar el título)
\end{sphinxadmonition}

\begin{sphinxadmonition}{tip}{Tip:}
\sphinxAtStartPar
This is a tip (no se puede cambiar el título)
\end{sphinxadmonition}


\sphinxstrong{See also:}
\nopagebreak


\sphinxAtStartPar
This is a seealso (no se puede cambiar el título)



\begin{sphinxadmonition}{note}{Note:}
\sphinxAtStartPar
The next info should be nested

\begin{sphinxadmonition}{warning}{Warning:}
\sphinxAtStartPar
Here’s my \sphinxstylestrong{warning}
\end{sphinxadmonition}
\end{sphinxadmonition}


\subsection{Cuadros (usando ::: :::)}
\label{\detokenize{docs/02_01_Cuadros:cuadros-usando}}
\sphinxAtStartPar
Para entornos donde no se reconoce la sintaxis
```
puede usarse la sintaxis
:::

\begin{sphinxadmonition}{important}{Important:}
\begin{sphinxadmonition}{note}{Note:}
\sphinxAtStartPar
Esto es una nota
\end{sphinxadmonition}
\end{sphinxadmonition}

\begin{sphinxadmonition}{note}{Cuadro de Warning}

\sphinxAtStartPar
This is a \sphinxstylestrong{warning}
\end{sphinxadmonition}


\subsection{Cuadros con html}
\label{\detokenize{docs/02_01_Cuadros:cuadros-con-html}}
\sphinxAtStartPar
A drawback of admonition syntax is that it will not render in interfaces that do not support this syntax (e.g., GitHub). If you’d like to use admonitions that are defined purely with HTML, MyST can parse them via the html\_admonitions extension.

\begin{sphinxadmonition}{note}{This is the \sphinxstylestrong{title}}

\sphinxAtStartPar
This is the \sphinxstyleemphasis{content}
\end{sphinxadmonition}

\sphinxAtStartPar
During the Sphinx render, both the class and name attributes will be used by Sphinx, but any other attributes like style will be discarded.

\sphinxAtStartPar
There can be no empty lines in the block, otherwise they will be read as two separate blocks. If you want to use multiple paragraphs then they can be enclosed in \sphinxcode{\sphinxupquote{<p>}}:

\begin{sphinxadmonition}{note}{Note}

\sphinxAtStartPar
Paragraph 1

\sphinxAtStartPar
Paragraph 2
\end{sphinxadmonition}

\begin{sphinxadmonition}{note}{Note}

\sphinxAtStartPar
Some \sphinxstylestrong{content}

\begin{sphinxadmonition}{note}{A \sphinxstyleemphasis{title}}

\sphinxAtStartPar
Paragraph 1

\sphinxAtStartPar
Paragraph 2
\end{sphinxadmonition}
\end{sphinxadmonition}

\sphinxstepscope
\begin{quote}

\sphinxAtStartPar
Dec 11, 2023 | 66 words | 0 min read
\end{quote}


\section{Footnotes}
\label{\detokenize{docs/02_02_Footnotes:footnotes}}\label{\detokenize{docs/02_02_Footnotes::doc}}
\sphinxAtStartPar
This is a manually\sphinxhyphen{}numbered footnote reference.%
\begin{footnote}[3]\sphinxAtStartFootnote
This is a manually\sphinxhyphen{}numbered footnote definition.
%
\end{footnote}

\sphinxAtStartPar
This is an auto\sphinxhyphen{}numbered footnote reference.%
\begin{footnote}[1]\sphinxAtStartFootnote
This is an auto\sphinxhyphen{}numbered footnote definition.
%
\end{footnote}

\sphinxAtStartPar
A longer footnote definition.%
\begin{footnote}[2]\sphinxAtStartFootnote
This is the \sphinxstyleemphasis{\sphinxstylestrong{footnote definition}}.
That continues for all indented lines
\begin{itemize}
\item {} 
\sphinxAtStartPar
even other block elements

\end{itemize}

\sphinxAtStartPar
Plus any preceding unindented lines, that are not separated by a blank line
%
\end{footnote}

\sphinxAtStartPar
Como podemos ver en la nota a pie de página \sphinxfootnotemark[2]


\bigskip\hrule\bigskip


\sphinxstepscope
\begin{quote}

\sphinxAtStartPar
Dec 11, 2023 | 31 words | 0 min read
\end{quote}


\section{Figuras}
\label{\detokenize{docs/02_03_Figuras:figuras}}\label{\detokenize{docs/02_03_Figuras::doc}}\begin{quote}

\sphinxAtStartPar
Align options: “top”, “middle”, “bottom”, “left”, “center”, or “right”
\end{quote}

\begin{figure}[htbp]
\centering
\capstart

\noindent\sphinxincludegraphics[scale=0.5]{{Descomp_ortogonal}.png}
\caption{This is the caption of the figure (a simple paragraph).}\label{\detokenize{docs/02_03_Figuras:fig-target}}\end{figure}

\sphinxAtStartPar
Referencia a la figura: \hyperref[\detokenize{docs/02_03_Figuras:fig-target}]{Fig.\@ \ref{\detokenize{docs/02_03_Figuras:fig-target}}}

\begin{figure}[htbp]
\centering
\capstart

\noindent{\hspace*{\fill}\sphinxincludegraphics[width=200\sphinxpxdimen]{{Descomp_ortogonal}.png}\hspace*{\fill}}
\caption{This is a caption in \sphinxstylestrong{Markdown}}\label{\detokenize{docs/02_03_Figuras:fig-target-2}}\end{figure}

\sphinxAtStartPar
Referencia a la figura: \hyperref[\detokenize{docs/02_03_Figuras:fig-target-2}]{Fig.\@ \ref{\detokenize{docs/02_03_Figuras:fig-target-2}}}

\sphinxAtStartPar
\sphinxincludegraphics{{logo-wide}.png}

\sphinxstepscope
\begin{quote}

\sphinxAtStartPar
Dec 11, 2023 | 334 words | 2 min read
\end{quote}


\section{Más cosas}
\label{\detokenize{docs/02_04_Mas_cosas:mas-cosas}}\label{\detokenize{docs/02_04_Mas_cosas::doc}}
\sphinxAtStartPar
Presentamos primero la linea horizontal, pues la usaremos (pueden ser tres \sphinxcode{\sphinxupquote{\sphinxhyphen{}}} o más):


\bigskip\hrule\bigskip

\begin{quote}

\sphinxAtStartPar
this is a quote. this is a quote. this is a quote. this is a quote. this is a quote. this is a quote

\sphinxAtStartPar
this is a quote

\sphinxAtStartPar
this is a quote
\end{quote}


\bigskip\hrule\bigskip


\sphinxAtStartPar
Veamos otra forma de hacer cuotas:
\begin{quote}

\sphinxAtStartPar
Here is a cool quotation.

\begin{flushright}
---Jo the Jovyan
\end{flushright}
\end{quote}


\subsection{Margin and sidebar}
\label{\detokenize{docs/02_04_Mas_cosas:margin-and-sidebar}}
\begin{sphinxShadowBox}
\sphinxstylesidebartitle{My sidebar title}

\sphinxAtStartPar
My sidebar content
\end{sphinxShadowBox}

\sphinxAtStartPar
If you use a sidebar within your content, the sidebar will stay in\sphinxhyphen{}line with your page’s content. However, it will be placed to the right, allowing your content to wrap around it. This prevents the sidebar from breaking up the flow of your content. This is particularly useful if you’ve got tall\sphinxhyphen{}and\sphinxhyphen{}long blocks of content or images that you would like to provide context to throughout your content.

\begin{sphinxShadowBox}
\sphinxstylesidebartitle{}

\sphinxAtStartPar
Escribimos en el margen !!!

\begin{sphinxadmonition}{note}{Note:}
\sphinxAtStartPar
Incluso escribimos notas!!!
\end{sphinxadmonition}
\end{sphinxShadowBox}


\subsection{hlist}
\label{\detokenize{docs/02_04_Mas_cosas:hlist}}
\sphinxAtStartPar
Veamos una hlist:
\begin{multicols}{3}\raggedright
\begin{itemize}\setlength{\itemsep}{0pt}\setlength{\parskip}{0pt}
\item {} 
\sphinxAtStartPar
Elemento 1

\item {} 
\sphinxAtStartPar
Elemento 2

\item {} 
\sphinxAtStartPar
Elemento 3

\item {} 
\sphinxAtStartPar
Elemento 4

\item {} 
\sphinxAtStartPar
Elemento 5

\item {} 
\sphinxAtStartPar
Elemento 6

\item {} 
\sphinxAtStartPar
Elemento 7

\end{itemize}\raggedcolumns\end{multicols}


\subsection{rubric}
\label{\detokenize{docs/02_04_Mas_cosas:rubric}}
\sphinxAtStartPar
Veamos un rubric:
\subsubsection*{This is a rubric (título chulo básicamente)}


\subsection{Centered}
\label{\detokenize{docs/02_04_Mas_cosas:centered}}
\sphinxAtStartPar
Veamos un centered:

\begin{center}Esto es un centered (negrita y centrado)
\end{center}

\subsection{Sectionauthor}
\label{\detokenize{docs/02_04_Mas_cosas:sectionauthor}}
\sphinxAtStartPar
Autor (no se como funciona):


\subsection{Glossary}
\label{\detokenize{docs/02_04_Mas_cosas:glossary}}
\sphinxAtStartPar
Veamos glosarios:
\begin{description}
\sphinxlineitem{Term one\index{Term one@\spxentry{Term one}|spxpagem}\phantomsection\label{\detokenize{docs/02_04_Mas_cosas:term-Term-one}}}
\sphinxAtStartPar
An indented explanation of term 1

\sphinxlineitem{A second term\index{A second term@\spxentry{A second term}|spxpagem}\phantomsection\label{\detokenize{docs/02_04_Mas_cosas:term-A-second-term}}}
\sphinxAtStartPar
An indented explanation of term2

\end{description}


\subsection{Sustitutions}
\label{\detokenize{docs/02_04_Mas_cosas:sustitutions}}
\sphinxAtStartPar
Veamos las sustitution (hay que añadirlas al encabezado de la página)

\sphinxAtStartPar
Sustitution 1: I’m a \sphinxstylestrong{substitution}

\sphinxAtStartPar
Sustitution 2: 
\begin{sphinxadmonition}{note}{Note:}
\sphinxAtStartPar
I’m a \sphinxstylestrong{substitution}
\end{sphinxadmonition}



\bigskip\hrule\bigskip


\sphinxAtStartPar
You can also define book\sphinxhyphen{}level substitution variables with the following configuration:

\begin{sphinxVerbatim}[commandchars=\\\{\}]
parse:
  myst\PYGZus{}substitutions:
    key: value
\end{sphinxVerbatim}


\bigskip\hrule\bigskip


\sphinxAtStartPar
Sustituciones con formato: MyST substitutions use Jinja templates in order to substitute in key / values. This means that you can apply any standard Jinja formatting to your substitutions. For example, you can replace text in your substitutions like so:

\sphinxAtStartPar
The original key1: I’m a \sphinxstylestrong{substitution}

\sphinxAtStartPar
I’m a \sphinxstylestrong{substitution}


\subsection{Grids}
\label{\detokenize{docs/02_04_Mas_cosas:grids}}
\begin{sphinxuseclass}{sd-container-fluid}
\begin{sphinxuseclass}{sd-sphinx-override}
\begin{sphinxuseclass}{sd-mb-4}
\begin{sphinxuseclass}{sd-row}
\begin{sphinxuseclass}{sd-g-4}
\begin{sphinxuseclass}{sd-g-xs-4}
\begin{sphinxuseclass}{sd-g-sm-4}
\begin{sphinxuseclass}{sd-g-md-4}
\begin{sphinxuseclass}{sd-g-lg-4}
\begin{sphinxuseclass}{sd-col}
\begin{sphinxuseclass}{sd-d-flex-column}
\begin{sphinxuseclass}{sd-border-1}
\sphinxAtStartPar
A

\end{sphinxuseclass}
\end{sphinxuseclass}
\end{sphinxuseclass}
\begin{sphinxuseclass}{sd-col}
\begin{sphinxuseclass}{sd-d-flex-column}
\begin{sphinxuseclass}{sd-border-1}
\sphinxAtStartPar
B

\end{sphinxuseclass}
\end{sphinxuseclass}
\end{sphinxuseclass}
\begin{sphinxuseclass}{sd-col}
\begin{sphinxuseclass}{sd-d-flex-column}
\begin{sphinxuseclass}{sd-border-1}
\sphinxAtStartPar
C

\end{sphinxuseclass}
\end{sphinxuseclass}
\end{sphinxuseclass}
\begin{sphinxuseclass}{sd-col}
\begin{sphinxuseclass}{sd-d-flex-column}
\begin{sphinxuseclass}{sd-border-1}
\sphinxAtStartPar
D

\end{sphinxuseclass}
\end{sphinxuseclass}
\end{sphinxuseclass}
\end{sphinxuseclass}
\end{sphinxuseclass}
\end{sphinxuseclass}
\end{sphinxuseclass}
\end{sphinxuseclass}
\end{sphinxuseclass}
\end{sphinxuseclass}
\end{sphinxuseclass}
\end{sphinxuseclass}

\bigskip\hrule\bigskip


\begin{sphinxuseclass}{sd-container-fluid}
\begin{sphinxuseclass}{sd-sphinx-override}
\begin{sphinxuseclass}{sd-mb-4}
\begin{sphinxuseclass}{sd-row}
\begin{sphinxuseclass}{sd-col}
\begin{sphinxuseclass}{sd-d-flex-column}
\begin{sphinxuseclass}{sd-col-3}
\begin{sphinxuseclass}{sd-col-xs-3}
\begin{sphinxuseclass}{sd-col-sm-3}
\begin{sphinxuseclass}{sd-col-md-3}
\begin{sphinxuseclass}{sd-col-lg-3}
\begin{sphinxuseclass}{sd-border-1}
\sphinxAtStartPar
A

\sphinxAtStartPar
A2

\end{sphinxuseclass}
\end{sphinxuseclass}
\end{sphinxuseclass}
\end{sphinxuseclass}
\end{sphinxuseclass}
\end{sphinxuseclass}
\end{sphinxuseclass}
\end{sphinxuseclass}
\begin{sphinxuseclass}{sd-col}
\begin{sphinxuseclass}{sd-d-flex-column}
\begin{sphinxuseclass}{sd-col-9}
\begin{sphinxuseclass}{sd-col-xs-9}
\begin{sphinxuseclass}{sd-col-sm-9}
\begin{sphinxuseclass}{sd-col-md-9}
\begin{sphinxuseclass}{sd-col-lg-9}
\begin{sphinxuseclass}{sd-border-1}
\sphinxAtStartPar
B

\sphinxAtStartPar
B2

\end{sphinxuseclass}
\end{sphinxuseclass}
\end{sphinxuseclass}
\end{sphinxuseclass}
\end{sphinxuseclass}
\end{sphinxuseclass}
\end{sphinxuseclass}
\end{sphinxuseclass}
\begin{sphinxuseclass}{sd-col}
\begin{sphinxuseclass}{sd-d-flex-column}
\begin{sphinxuseclass}{sd-col-6}
\begin{sphinxuseclass}{sd-col-xs-6}
\begin{sphinxuseclass}{sd-col-sm-6}
\begin{sphinxuseclass}{sd-col-md-6}
\begin{sphinxuseclass}{sd-col-lg-6}
\begin{sphinxuseclass}{sd-border-1}
\sphinxAtStartPar
C

\sphinxAtStartPar
C2

\end{sphinxuseclass}
\end{sphinxuseclass}
\end{sphinxuseclass}
\end{sphinxuseclass}
\end{sphinxuseclass}
\end{sphinxuseclass}
\end{sphinxuseclass}
\end{sphinxuseclass}
\begin{sphinxuseclass}{sd-col}
\begin{sphinxuseclass}{sd-d-flex-column}
\begin{sphinxuseclass}{sd-col-6}
\begin{sphinxuseclass}{sd-col-xs-6}
\begin{sphinxuseclass}{sd-col-sm-6}
\begin{sphinxuseclass}{sd-col-md-6}
\begin{sphinxuseclass}{sd-col-lg-6}
\begin{sphinxuseclass}{sd-border-1}
\sphinxAtStartPar
D

\sphinxAtStartPar
D2

\end{sphinxuseclass}
\end{sphinxuseclass}
\end{sphinxuseclass}
\end{sphinxuseclass}
\end{sphinxuseclass}
\end{sphinxuseclass}
\end{sphinxuseclass}
\end{sphinxuseclass}
\end{sphinxuseclass}
\end{sphinxuseclass}
\end{sphinxuseclass}
\end{sphinxuseclass}

\bigskip\hrule\bigskip


\begin{sphinxuseclass}{sd-container-fluid}
\begin{sphinxuseclass}{sd-sphinx-override}
\begin{sphinxuseclass}{sd-mb-4}
\begin{sphinxuseclass}{sd-row}
\begin{sphinxuseclass}{sd-g-3}
\begin{sphinxuseclass}{sd-g-xs-3}
\begin{sphinxuseclass}{sd-g-sm-3}
\begin{sphinxuseclass}{sd-g-md-3}
\begin{sphinxuseclass}{sd-g-lg-3}
\begin{sphinxuseclass}{sd-col}
\begin{sphinxuseclass}{sd-d-flex-row}
\begin{sphinxuseclass}{sd-card}
\begin{sphinxuseclass}{sd-sphinx-override}
\begin{sphinxuseclass}{sd-w-100}
\begin{sphinxuseclass}{sd-shadow-sm}
\begin{sphinxuseclass}{sd-card-body}
\begin{sphinxuseclass}{sd-card-title}
\begin{sphinxuseclass}{sd-font-weight-bold}One!
\end{sphinxuseclass}
\end{sphinxuseclass}
\sphinxAtStartPar
Here’s the first card.

\end{sphinxuseclass}
\end{sphinxuseclass}
\end{sphinxuseclass}
\end{sphinxuseclass}
\end{sphinxuseclass}
\end{sphinxuseclass}
\end{sphinxuseclass}
\begin{sphinxuseclass}{sd-col}
\begin{sphinxuseclass}{sd-d-flex-row}
\begin{sphinxuseclass}{sd-card}
\begin{sphinxuseclass}{sd-sphinx-override}
\begin{sphinxuseclass}{sd-w-100}
\begin{sphinxuseclass}{sd-shadow-sm}
\begin{sphinxuseclass}{sd-card-body}
\begin{sphinxuseclass}{sd-card-title}
\begin{sphinxuseclass}{sd-font-weight-bold}Two!
\end{sphinxuseclass}
\end{sphinxuseclass}
\sphinxAtStartPar
Here’s the second card.

\end{sphinxuseclass}
\end{sphinxuseclass}
\end{sphinxuseclass}
\end{sphinxuseclass}
\end{sphinxuseclass}
\end{sphinxuseclass}
\end{sphinxuseclass}
\begin{sphinxuseclass}{sd-col}
\begin{sphinxuseclass}{sd-d-flex-row}
\begin{sphinxuseclass}{sd-card}
\begin{sphinxuseclass}{sd-sphinx-override}
\begin{sphinxuseclass}{sd-w-100}
\begin{sphinxuseclass}{sd-shadow-sm}
\begin{sphinxuseclass}{sd-card-body}
\begin{sphinxuseclass}{sd-card-title}
\begin{sphinxuseclass}{sd-font-weight-bold}Three!
\end{sphinxuseclass}
\end{sphinxuseclass}
\sphinxAtStartPar
Here’s the third card.

\end{sphinxuseclass}
\end{sphinxuseclass}
\end{sphinxuseclass}
\end{sphinxuseclass}
\end{sphinxuseclass}
\end{sphinxuseclass}
\end{sphinxuseclass}
\end{sphinxuseclass}
\end{sphinxuseclass}
\end{sphinxuseclass}
\end{sphinxuseclass}
\end{sphinxuseclass}
\end{sphinxuseclass}
\end{sphinxuseclass}
\end{sphinxuseclass}
\end{sphinxuseclass}

\subsection{Dropdowns}
\label{\detokenize{docs/02_04_Mas_cosas:dropdowns}}\subsubsection*{Here’s my dropdown}

\sphinxAtStartPar
And here’s my dropdown content

\begin{sphinxadmonition}{note}{Click here!}

\sphinxAtStartPar
Here’s what’s inside!
\end{sphinxadmonition}

\begin{sphinxadmonition}{note}{Note:}
\sphinxAtStartPar
The note body will be hidden!
\end{sphinxadmonition}


\subsection{Tab content}
\label{\detokenize{docs/02_04_Mas_cosas:tab-content}}
\begin{sphinxuseclass}{sd-tab-set}
\begin{sphinxuseclass}{sd-tab-item}\subsubsection*{Tab 1 title}

\begin{sphinxuseclass}{sd-tab-content}
\sphinxAtStartPar
My first tab

\end{sphinxuseclass}
\end{sphinxuseclass}
\begin{sphinxuseclass}{sd-tab-item}\subsubsection*{Tab 2 title}

\begin{sphinxuseclass}{sd-tab-content}
\sphinxAtStartPar
My second tab with \sphinxcode{\sphinxupquote{some code}}!

\end{sphinxuseclass}
\end{sphinxuseclass}
\end{sphinxuseclass}
\begin{sphinxuseclass}{sd-tab-set}
\begin{sphinxuseclass}{sd-tab-item}\subsubsection*{C++}

\begin{sphinxuseclass}{sd-tab-content}
\begin{sphinxVerbatim}[commandchars=\\\{\}]
\PYG{k+kt}{int}\PYG{+w}{ }\PYG{n+nf}{main}\PYG{p}{(}\PYG{k}{const}\PYG{+w}{ }\PYG{k+kt}{int}\PYG{+w}{ }\PYG{n}{argc}\PYG{p}{,}\PYG{+w}{ }\PYG{k}{const}\PYG{+w}{ }\PYG{k+kt}{char}\PYG{+w}{ }\PYG{o}{*}\PYG{o}{*}\PYG{n}{argv}\PYG{p}{)}\PYG{+w}{ }\PYG{p}{\PYGZob{}}
\PYG{+w}{  }\PYG{k}{return}\PYG{+w}{ }\PYG{l+m+mi}{0}\PYG{p}{;}
\PYG{p}{\PYGZcb{}}
\end{sphinxVerbatim}

\end{sphinxuseclass}
\end{sphinxuseclass}
\begin{sphinxuseclass}{sd-tab-item}\subsubsection*{Python}

\begin{sphinxuseclass}{sd-tab-content}
\begin{sphinxVerbatim}[commandchars=\\\{\}]
\PYG{k}{def} \PYG{n+nf}{main}\PYG{p}{(}\PYG{p}{)}\PYG{p}{:}
    \PYG{k}{return}
\end{sphinxVerbatim}

\end{sphinxuseclass}
\end{sphinxuseclass}
\begin{sphinxuseclass}{sd-tab-item}\subsubsection*{Java}

\begin{sphinxuseclass}{sd-tab-content}
\begin{sphinxVerbatim}[commandchars=\\\{\}]
\PYG{k+kd}{class} \PYG{n+nc}{Main} \PYG{p}{\PYGZob{}}
    \PYG{k+kd}{public} \PYG{k+kd}{static} \PYG{k+kt}{void} \PYG{n+nf}{main}\PYG{p}{(}\PYG{n}{String}\PYG{o}{[}\PYG{o}{]} \PYG{n}{args}\PYG{p}{)} \PYG{p}{\PYGZob{}}
    \PYG{p}{\PYGZcb{}}
\PYG{p}{\PYGZcb{}}
\end{sphinxVerbatim}

\end{sphinxuseclass}
\end{sphinxuseclass}
\end{sphinxuseclass}

\subsection{Table}
\label{\detokenize{docs/02_04_Mas_cosas:table}}

\begin{savenotes}\sphinxattablestart
\centering
\sphinxcapstartof{table}
\sphinxthecaptionisattop
\sphinxcaption{My table title}\label{\detokenize{docs/02_04_Mas_cosas:my-table-ref}}
\sphinxaftertopcaption
\begin{tabulary}{\linewidth}[t]{|T|T|}
\hline
\sphinxstyletheadfamily 
\sphinxAtStartPar
header 1
&\sphinxstyletheadfamily 
\sphinxAtStartPar
header 2
\\
\hline
\sphinxAtStartPar
3
&
\sphinxAtStartPar
4
\\
\hline
\end{tabulary}
\par
\sphinxattableend\end{savenotes}

\sphinxAtStartPar
Here is \hyperref[\detokenize{docs/02_04_Mas_cosas:my-table-ref}]{Table \ref{\detokenize{docs/02_04_Mas_cosas:my-table-ref}}}

\sphinxstepscope
\begin{quote}

\sphinxAtStartPar
Dec 11, 2023 | 73 words | 0 min read
\end{quote}


\chapter{Referenciar cosas}
\label{\detokenize{docs/03_00_referenciar_cosas:referenciar-cosas}}\label{\detokenize{docs/03_00_referenciar_cosas::doc}}
\sphinxAtStartPar
Referencia a una sección usando el título: {\hyperref[\detokenize{docs/01_00_Code_Blocks_y_ecuaciones::doc}]{\sphinxcrossref{\DUrole{doc}{Code Blocks y ecuaciones}}}}. Se usa el \sphinxstylestrong{nombre del fichero}

\begin{sphinxVerbatim}[commandchars=\\\{\}]
\PYGZob{}doc\PYGZcb{}`./01\PYGZus{}00\PYGZus{}sec\PYGZus{}Code\PYGZus{}Blocks\PYGZus{}y\PYGZus{}Ecuaciones`
\end{sphinxVerbatim}

\sphinxAtStartPar
Referenciamos una sección \hyperref[\detokenize{docs/01_00_Code_Blocks_y_ecuaciones:sec-code-blocks-y-ecuaciones}]{Section \ref{\detokenize{docs/01_00_Code_Blocks_y_ecuaciones:sec-code-blocks-y-ecuaciones}}}. Se usa una \sphinxstylestrong{label}

\begin{sphinxVerbatim}[commandchars=\\\{\}]
`(sec\PYGZus{}Code\PYGZus{}Blocks\PYGZus{}y\PYGZus{}Ecuaciones)=`     
\PYGZsh{} Code Blocks y Ecuaciones 

\PYGZob{}numref\PYGZcb{}`sec\PYGZus{}Code\PYGZus{}Blocks\PYGZus{}y\PYGZus{}Ecuaciones`.
\end{sphinxVerbatim}

\sphinxAtStartPar
Referencias a ecuaciones: \eqref{equation:docs/01_02_Ecuaciones:my_other_label}

\begin{sphinxVerbatim}[commandchars=\\\{\}]
\PYGZob{}eq\PYGZcb{}`my\PYGZus{}other\PYGZus{}label`
\end{sphinxVerbatim}

\sphinxAtStartPar
Referencias a figuras: \hyperref[\detokenize{docs/02_03_Figuras:fig-target}]{Fig.\@ \ref{\detokenize{docs/02_03_Figuras:fig-target}}}

\begin{sphinxVerbatim}[commandchars=\\\{\}]
\PYGZob{}numref\PYGZcb{}`fig\PYGZhy{}target`
\end{sphinxVerbatim}

\sphinxAtStartPar
Referencias a bloques de código \hyperref[\detokenize{docs/01_01_Code_Blocks:label-codeblock}]{Listing \ref{\detokenize{docs/01_01_Code_Blocks:label-codeblock}}} (deben de tener tanto :caption: como :name:)

\begin{sphinxVerbatim}[commandchars=\\\{\}]
\PYGZob{}numref\PYGZcb{}`label\PYGZus{}codeblock`
\end{sphinxVerbatim}


\section{Editar nombre en numref:}
\label{\detokenize{docs/03_00_referenciar_cosas:editar-nombre-en-numref}}
\sphinxAtStartPar
Podemos editar el nombre que apare en las numref antes del número. Por ejemplo, podemos pasar de \hyperref[\detokenize{docs/02_03_Figuras:fig-target}]{Fig.\@ \ref{\detokenize{docs/02_03_Figuras:fig-target}}} a \hyperref[\detokenize{docs/02_03_Figuras:fig-target}]{Figura \ref{\detokenize{docs/02_03_Figuras:fig-target}}}.

\begin{sphinxVerbatim}[commandchars=\\\{\}]
\PYGZob{}numref\PYGZcb{}`Fig. \PYGZpc{}s \PYGZlt{}fig\PYGZhy{}target\PYGZgt{}`
\end{sphinxVerbatim}

\sphinxAtStartPar
Otro ejemplo: \hyperref[\detokenize{docs/01_00_Code_Blocks_y_ecuaciones:sec-code-blocks-y-ecuaciones}]{sec.\@ \ref{\detokenize{docs/01_00_Code_Blocks_y_ecuaciones:sec-code-blocks-y-ecuaciones}}} en vez de \hyperref[\detokenize{docs/01_00_Code_Blocks_y_ecuaciones:sec-code-blocks-y-ecuaciones}]{Section \ref{\detokenize{docs/01_00_Code_Blocks_y_ecuaciones:sec-code-blocks-y-ecuaciones}}}.

\begin{sphinxVerbatim}[commandchars=\\\{\}]
\PYGZob{}numref\PYGZcb{}`sec. \PYGZpc{}s \PYGZlt{}sec\PYGZus{}Code\PYGZus{}Blocks\PYGZus{}y\PYGZus{}Ecuaciones\PYGZgt{}`
\end{sphinxVerbatim}


\section{Referencias bibliográficas}
\label{\detokenize{docs/03_00_referenciar_cosas:referencias-bibliograficas}}
\sphinxAtStartPar
Cita: \sphinxcode{\sphinxupquote{\{cite\}`guttag2016introduction`}}: {[}\hyperlink{cite.docs/03_00_referenciar_cosas:id3}{1}{]}

\sphinxstepscope
\begin{quote}

\sphinxAtStartPar
Dec 11, 2023 | 167 words | 1 min read
\end{quote}


\chapter{Teoremas, pruebas, algoritmos …}
\label{\detokenize{docs/04_00_Teoremas_pruebas_y_algoritmos:teoremas-pruebas-algoritmos}}\label{\detokenize{docs/04_00_Teoremas_pruebas_y_algoritmos::doc}}
\sphinxAtStartPar
Infrastructure to support items such as proof and algorithm style formatting is provided by the \sphinxhref{https://sphinx-proof.readthedocs.io/en/latest/}{sphinx\sphinxhyphen{}proof} extension.

\sphinxAtStartPar
Para ver todas las directivas: \sphinxurl{https://sphinx-proof.readthedocs.io/en/latest/syntax.html\#collection-of-directives}


\section{Theorems}
\label{\detokenize{docs/04_00_Teoremas_pruebas_y_algoritmos:theorems}}\label{docs/04_00_Teoremas_pruebas_y_algoritmos:my-theorem}
\begin{sphinxadmonition}{note}{Theorem 4.1 (Titulo del teorema (opcional))}



\sphinxAtStartPar
Esto sería un teorema
\end{sphinxadmonition}

\sphinxAtStartPar
Referenciamos: {\hyperref[\detokenize{docs/04_00_Teoremas_pruebas_y_algoritmos:my-theorem}]{\sphinxcrossref{Theorem 4.1}}}


\section{Lemmas}
\label{\detokenize{docs/04_00_Teoremas_pruebas_y_algoritmos:lemmas}}\label{docs/04_00_Teoremas_pruebas_y_algoritmos:my-lemma}
\begin{sphinxadmonition}{note}{Lemma 4.1 (Titulo del lemma (opcional))}



\sphinxAtStartPar
Esto sería un lemma
\end{sphinxadmonition}

\sphinxAtStartPar
Referenciamos: {\hyperref[\detokenize{docs/04_00_Teoremas_pruebas_y_algoritmos:my-lemma}]{\sphinxcrossref{Lemma 4.1}}}


\section{Corollaries}
\label{\detokenize{docs/04_00_Teoremas_pruebas_y_algoritmos:corollaries}}\label{docs/04_00_Teoremas_pruebas_y_algoritmos:my-corollary}
\begin{sphinxadmonition}{note}{Corollary 4.1 (Titulo del corollary (opcional))}



\sphinxAtStartPar
Esto sería un corollary
\end{sphinxadmonition}

\sphinxAtStartPar
Referenciamos: {\hyperref[\detokenize{docs/04_00_Teoremas_pruebas_y_algoritmos:my-corollary}]{\sphinxcrossref{Corollary 4.1}}}


\section{Proofs}
\label{\detokenize{docs/04_00_Teoremas_pruebas_y_algoritmos:proofs}}
\begin{sphinxadmonition}{note}
\sphinxAtStartPar
Proof.
Esto sería un proof. No se puede referenciar
\end{sphinxadmonition}


\section{Definitions}
\label{\detokenize{docs/04_00_Teoremas_pruebas_y_algoritmos:definitions}}\label{docs/04_00_Teoremas_pruebas_y_algoritmos:my-definition}
\begin{sphinxadmonition}{note}{Definition 4.1 (Titulo del definition (opcional))}



\sphinxAtStartPar
Esto sería un definition
\end{sphinxadmonition}

\sphinxAtStartPar
Referenciamos: {\hyperref[\detokenize{docs/04_00_Teoremas_pruebas_y_algoritmos:my-definition}]{\sphinxcrossref{Definition 4.1}}}


\section{Examples}
\label{\detokenize{docs/04_00_Teoremas_pruebas_y_algoritmos:examples}}\label{docs/04_00_Teoremas_pruebas_y_algoritmos:my-example}
\begin{sphinxadmonition}{note}{Example 4.1 (Titulo del example (opcional))}



\sphinxAtStartPar
Esto sería un example
\end{sphinxadmonition}

\sphinxAtStartPar
Referenciamos: {\hyperref[\detokenize{docs/04_00_Teoremas_pruebas_y_algoritmos:my-example}]{\sphinxcrossref{Example 4.1}}}


\section{Axioms}
\label{\detokenize{docs/04_00_Teoremas_pruebas_y_algoritmos:axioms}}\label{docs/04_00_Teoremas_pruebas_y_algoritmos:my-axiom}
\begin{sphinxadmonition}{note}{Axiom 4.1 (Titulo del axiom (opcional))}



\sphinxAtStartPar
Esto sería un axiom
\end{sphinxadmonition}

\sphinxAtStartPar
Referenciamos: {\hyperref[\detokenize{docs/04_00_Teoremas_pruebas_y_algoritmos:my-axiom}]{\sphinxcrossref{Axiom 4.1}}}


\section{Algoritms}
\label{\detokenize{docs/04_00_Teoremas_pruebas_y_algoritmos:algoritms}}\label{docs/04_00_Teoremas_pruebas_y_algoritmos:my-algorithm}
\begin{sphinxadmonition}{note}{Algorithm 4.1 (Titulo del algoritm (opcional))}



\sphinxAtStartPar
Esto sería un algorithm
\end{sphinxadmonition}

\sphinxAtStartPar
Referenciamos: {\hyperref[\detokenize{docs/04_00_Teoremas_pruebas_y_algoritmos:my-algorithm}]{\sphinxcrossref{Algorithm 4.1}}}


\section{Conjectures}
\label{\detokenize{docs/04_00_Teoremas_pruebas_y_algoritmos:conjectures}}\label{docs/04_00_Teoremas_pruebas_y_algoritmos:my-conjecture}
\begin{sphinxadmonition}{note}{Conjecture 4.1 (Titulo del conjetures (opcional))}



\sphinxAtStartPar
Esto sería un conjetures
\end{sphinxadmonition}

\sphinxAtStartPar
Referenciamos: {\hyperref[\detokenize{docs/04_00_Teoremas_pruebas_y_algoritmos:my-conjecture}]{\sphinxcrossref{Conjecture 4.1}}}


\section{Criteria}
\label{\detokenize{docs/04_00_Teoremas_pruebas_y_algoritmos:criteria}}\label{docs/04_00_Teoremas_pruebas_y_algoritmos:my-criteria}
\begin{sphinxadmonition}{note}{Criterion 4.1 (Titulo del criteria (opcional))}



\sphinxAtStartPar
Esto sería un criteria
\end{sphinxadmonition}

\sphinxAtStartPar
Referenciamos: {\hyperref[\detokenize{docs/04_00_Teoremas_pruebas_y_algoritmos:my-criteria}]{\sphinxcrossref{Criterion 4.1}}}


\section{Observations}
\label{\detokenize{docs/04_00_Teoremas_pruebas_y_algoritmos:observations}}\label{docs/04_00_Teoremas_pruebas_y_algoritmos:my-observation}
\begin{sphinxadmonition}{note}{Observation 4.1 (Titulo del observation (opcional))}



\sphinxAtStartPar
Esto sería un observation
\end{sphinxadmonition}

\sphinxAtStartPar
Referenciamos: {\hyperref[\detokenize{docs/04_00_Teoremas_pruebas_y_algoritmos:my-observation}]{\sphinxcrossref{Observation 4.1}}}


\section{Properties}
\label{\detokenize{docs/04_00_Teoremas_pruebas_y_algoritmos:properties}}\label{docs/04_00_Teoremas_pruebas_y_algoritmos:my-property}
\begin{sphinxadmonition}{note}{Property 4.1 (Titulo del property (opcional))}



\sphinxAtStartPar
Esto sería un property
\end{sphinxadmonition}

\sphinxAtStartPar
Referenciamos: {\hyperref[\detokenize{docs/04_00_Teoremas_pruebas_y_algoritmos:my-property}]{\sphinxcrossref{Property 4.1}}}


\section{Propositions}
\label{\detokenize{docs/04_00_Teoremas_pruebas_y_algoritmos:propositions}}\label{docs/04_00_Teoremas_pruebas_y_algoritmos:my-proposition}
\begin{sphinxadmonition}{note}{Proposition 4.1 (Titulo del proposition (opcional))}



\sphinxAtStartPar
Esto sería un proposition
\end{sphinxadmonition}

\sphinxAtStartPar
Referenciamos: {\hyperref[\detokenize{docs/04_00_Teoremas_pruebas_y_algoritmos:my-proposition}]{\sphinxcrossref{Proposition 4.1}}}


\section{Remarks}
\label{\detokenize{docs/04_00_Teoremas_pruebas_y_algoritmos:remarks}}\label{docs/04_00_Teoremas_pruebas_y_algoritmos:my-remark}
\begin{sphinxadmonition}{note}{Remark 4.1 (Titulo del remark (opcional))}



\sphinxAtStartPar
Esto sería un remarks
\end{sphinxadmonition}

\sphinxAtStartPar
Referenciamos: {\hyperref[\detokenize{docs/04_00_Teoremas_pruebas_y_algoritmos:my-remark}]{\sphinxcrossref{Remark 4.1}}}

\begin{sphinxthebibliography}{1}
\bibitem[1]{docs/03_00_referenciar_cosas:id3}
\sphinxAtStartPar
John Guttag. \sphinxstyleemphasis{Introduction to computation and programming using Python: With application to understanding data}. MIT Press, 2016.
\end{sphinxthebibliography}






\renewcommand{\indexname}{Proof Index}
\begin{sphinxtheindex}
\let\bigletter\sphinxstyleindexlettergroup
\bigletter{my\sphinxhyphen{}algorithm}
\item\relax\sphinxstyleindexentry{my\sphinxhyphen{}algorithm}\sphinxstyleindexextra{docs/04\_00\_Teoremas\_pruebas\_y\_algoritmos}\sphinxstyleindexpageref{docs/04_00_Teoremas_pruebas_y_algoritmos:\detokenize{my-algorithm}}
\indexspace
\bigletter{my\sphinxhyphen{}axiom}
\item\relax\sphinxstyleindexentry{my\sphinxhyphen{}axiom}\sphinxstyleindexextra{docs/04\_00\_Teoremas\_pruebas\_y\_algoritmos}\sphinxstyleindexpageref{docs/04_00_Teoremas_pruebas_y_algoritmos:\detokenize{my-axiom}}
\indexspace
\bigletter{my\sphinxhyphen{}conjecture}
\item\relax\sphinxstyleindexentry{my\sphinxhyphen{}conjecture}\sphinxstyleindexextra{docs/04\_00\_Teoremas\_pruebas\_y\_algoritmos}\sphinxstyleindexpageref{docs/04_00_Teoremas_pruebas_y_algoritmos:\detokenize{my-conjecture}}
\indexspace
\bigletter{my\sphinxhyphen{}corollary}
\item\relax\sphinxstyleindexentry{my\sphinxhyphen{}corollary}\sphinxstyleindexextra{docs/04\_00\_Teoremas\_pruebas\_y\_algoritmos}\sphinxstyleindexpageref{docs/04_00_Teoremas_pruebas_y_algoritmos:\detokenize{my-corollary}}
\indexspace
\bigletter{my\sphinxhyphen{}criteria}
\item\relax\sphinxstyleindexentry{my\sphinxhyphen{}criteria}\sphinxstyleindexextra{docs/04\_00\_Teoremas\_pruebas\_y\_algoritmos}\sphinxstyleindexpageref{docs/04_00_Teoremas_pruebas_y_algoritmos:\detokenize{my-criteria}}
\indexspace
\bigletter{my\sphinxhyphen{}definition}
\item\relax\sphinxstyleindexentry{my\sphinxhyphen{}definition}\sphinxstyleindexextra{docs/04\_00\_Teoremas\_pruebas\_y\_algoritmos}\sphinxstyleindexpageref{docs/04_00_Teoremas_pruebas_y_algoritmos:\detokenize{my-definition}}
\indexspace
\bigletter{my\sphinxhyphen{}example}
\item\relax\sphinxstyleindexentry{my\sphinxhyphen{}example}\sphinxstyleindexextra{docs/04\_00\_Teoremas\_pruebas\_y\_algoritmos}\sphinxstyleindexpageref{docs/04_00_Teoremas_pruebas_y_algoritmos:\detokenize{my-example}}
\indexspace
\bigletter{my\sphinxhyphen{}lemma}
\item\relax\sphinxstyleindexentry{my\sphinxhyphen{}lemma}\sphinxstyleindexextra{docs/04\_00\_Teoremas\_pruebas\_y\_algoritmos}\sphinxstyleindexpageref{docs/04_00_Teoremas_pruebas_y_algoritmos:\detokenize{my-lemma}}
\indexspace
\bigletter{my\sphinxhyphen{}observation}
\item\relax\sphinxstyleindexentry{my\sphinxhyphen{}observation}\sphinxstyleindexextra{docs/04\_00\_Teoremas\_pruebas\_y\_algoritmos}\sphinxstyleindexpageref{docs/04_00_Teoremas_pruebas_y_algoritmos:\detokenize{my-observation}}
\indexspace
\bigletter{my\sphinxhyphen{}property}
\item\relax\sphinxstyleindexentry{my\sphinxhyphen{}property}\sphinxstyleindexextra{docs/04\_00\_Teoremas\_pruebas\_y\_algoritmos}\sphinxstyleindexpageref{docs/04_00_Teoremas_pruebas_y_algoritmos:\detokenize{my-property}}
\indexspace
\bigletter{my\sphinxhyphen{}proposition}
\item\relax\sphinxstyleindexentry{my\sphinxhyphen{}proposition}\sphinxstyleindexextra{docs/04\_00\_Teoremas\_pruebas\_y\_algoritmos}\sphinxstyleindexpageref{docs/04_00_Teoremas_pruebas_y_algoritmos:\detokenize{my-proposition}}
\indexspace
\bigletter{my\sphinxhyphen{}remark}
\item\relax\sphinxstyleindexentry{my\sphinxhyphen{}remark}\sphinxstyleindexextra{docs/04\_00\_Teoremas\_pruebas\_y\_algoritmos}\sphinxstyleindexpageref{docs/04_00_Teoremas_pruebas_y_algoritmos:\detokenize{my-remark}}
\indexspace
\bigletter{my\sphinxhyphen{}theorem}
\item\relax\sphinxstyleindexentry{my\sphinxhyphen{}theorem}\sphinxstyleindexextra{docs/04\_00\_Teoremas\_pruebas\_y\_algoritmos}\sphinxstyleindexpageref{docs/04_00_Teoremas_pruebas_y_algoritmos:\detokenize{my-theorem}}
\end{sphinxtheindex}

\renewcommand{\indexname}{Index}
\printindex
\end{document}